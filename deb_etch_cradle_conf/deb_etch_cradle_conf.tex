%
% Copyright (c) 2004, 2005, 2006 Andrew De Ponte.
% Permission is granted to copy, distribute and/or modify this document
% under the terms of the GNU Free Documentation License, Version 1.2
% or any later version published by the Free Software Foundation;
% with the Invariant Sections being Front-Cover Texts, with the
% Front-Cover Texts being abstract. A copy of the license is included
% in the section entitled "GNU Free Documentation License".
%

%
% FileName:     deb_etch_cradle_conf.tex
% Author:       Andrew De Ponte
% E-Mail:       cyphactor@socal.rr.com
% AIM:          HUNNYnNUTTS
% Description:  This document is a LaTeX document containing the steps which
%               I took to configure a stock Debian etch Linux system to handle
%               the synchronization of my Zaurus SL-5600.
%

% Select the document class
\documentclass{article}

\usepackage{fullpage}

% Set the document information
\title{Zaurus Debian "etch (testing)" Cradle Synchronization Configuration}
\author{Andrew De Ponte}
\date{\today}

% Begin the document
\begin{document}

% Now that i have set the information, I want to create the title of the
% document.
\maketitle

% Here I place a page break so that the abstract has its own space.
\newpage

% Here I put the abstract of the document so the readers can decide if this
% is the correct document to read.
\begin{abstract}
This document outlines the means used to configure a Debian etch
(testing) system to handle synchronization of a Zaurus using the cradle
and the synchronization software provided by
http://zsrep.sourceforge.net.
\end{abstract}

% Here I place a page break so that the table of contents starts on its own
% page.
\newpage

% Here I place the copyright notice and license information in which we
% state that this document falls under the GNU FDL or GNU Free Documentation
% License.
\noindent Copyright \copyright 2004, 2005, 2006 Andrew De Ponte.\\
Permission is granted to copy, distribute and/or modify this document under
the terms of the GNU Free Documentation License, Version 1.2 or any later
version published by the Free Software Foundation; with the Invariant
Sections being Front-Cover Texts, with the Front-Cover  Texts being
abstract. A copy of the license is included in the section entitled "GNU
Free Documentation License".

% Here I place a page break so that the license information does not get
% combined with the table of contents.
\newpage

% Place the table of contents of the document here.
\tableofcontents

% Here I place a page break so that the table of contents exists in its own
% space and does not fall into the actual document space.
\newpage

% Here should be the actual content of the document.

\section{Required Driver}

The stock Debian etch kernel 2.6.16 has built-in driver support for the
Zaurus hence you need not do anything if you are using this kernel.  The
driver (kernel module) which is used for the Zaurus is the \emph{usbnet}
driver.

\section{Network Configuration}

The Zaurus driver (\emph{usbnet}) is actually a driver which allows a
USB device to act as a network device. The power of this is that it
allows applications written for standard network protocols to work over
the USB connection between the Zaurus and the computer. Since this
driver basically creates a network device, one has to setup a
configuration file so that the network device is automatically brought
up and configured by system.  This driver (\emph{usbnet}) creates a USB
Ethernet device, normally named (\emph{usb0}).  This device needs to
have the address of \emph{192.168.129.1}, a class C network address, due
to the fact that the address of the Zaurus is \emph{192.168.129.201}, a
class C network address.

    \subsection{/etc/network/interfaces}

    The method for configuring network devices in debian is to add an
    entry to the \emph{/etc/network/interfaces} configuration file. The
    \emph{/etc/network/interfaces} file should contain the following:

    \begin{verbatim}
    auto usb0
    allow-hotplug usb0
    iface usb0 inet static
        address 192.168.129.1
        netmask 255.255.255.0
        network 192.168.129.0
        broadcast 192.168.129.255
    \end{verbatim}

    \subsection{Zaurus PC Link}

    On the Zaurus end of things, the Zaurus must have its IP address set as
    well. This is done on Sharp ROMs by using their \emph{PC Link}
    application. Go to the \emph{PC Link} application on your Zaurus. It can
    be found on your Zaurus under the \emph{Settings} tab. Once in the
    \emph{PC Link} application, you want to make sure the Zaurus hostname is
    set to ``zaurus'' (without the quotes) and that the USB IP address of the
    Zaurus is set to ``192.168.129.201'' (without the quotes). You also want
    to make sure that the proper connection type is selected. The \emph{USB -
    TCP/IP (advanced)} option, should be the connection option that is
    selected in the drop down list for the connection type. If it is
    \emph{NOT} then select it. After, the above have been performed tap the
    \emph{OK} button to apply your changes.

\section{Zaurus and Desktop Meet}

It is now time for the Zaurus and the Debian Desktop to meet.

\subsection{Watching the Log File}

Now, one will want to know that everything is working properly. Hence, they
would want to see the log output of what ever they are doing. In this case one
wants to see the log output of the kernel, kernel modules, and hotplug
system. To do so one would run the following command in a terminal as super
user:

\begin{verbatim}
# tail -f /var/log/messages
\end{verbatim}

Keep this terminal open and visible throughout the following process for it
will display changes made to log file as things are performed.

\subsection{Plugging in the Cradle}

At this stage one should plug in their \emph{empty} Zaurus cradle into a
working USB port on their Debian box. Note: You will not see anything happen
in the \emph{/var/log/messages} tail yet because the cradle isn't the device,
the cradle is basically just a spiffy wire that holds the
Zaurus rather than just connecting the Zaurus.

\subsection{Plugging in the Zaurus}

At this stage one should take their Zaurus, \emph{in powered off state}, and
place it fully into the cradle. Note: You still won't see anything in the
\emph{/var/log/messages} tail yet because the Zaurus is off.

\subsection{Turning on the Zaurus}

Turn on your Zaurus, leaving it in the cradle. You should see in the
\emph{/var/log/messages} tail something similar to the following:

\begin{verbatim}
Sep 12 13:26:13 devtux kernel: usb 1-1: new full speed USB device using ohci_hcd and address 3
Sep 12 13:26:13 devtux kernel: usb 1-1: configuration #1 chosen from 1 choice
Sep 12 13:26:13 devtux kernel: usb0: register 'zaurus' at usb-0000:00:02.0-1, pseudo-MDLM (BLAN) device, 82:1e:3b:d9:c9:23
\end{verbatim}

\subsection{Checking Configuration}

At this point you should now run the following as super user to check if the
driver was loaded properly and the device configured properly:

\begin{verbatim}
# /sbin/ifconfig usb0
\end{verbatim}

You should see the following:

\begin{verbatim}
usb0      Link encap:Ethernet  HWaddr 82:1E:3B:D9:C9:23  
          inet addr:192.168.129.1  Bcast:192.168.129.255  Mask:255.255.255.0
          inet6 addr: fe80::801e:3bff:fed9:c923/64 Scope:Link
          UP BROADCAST RUNNING MULTICAST  MTU:1500  Metric:1
          RX packets:21 errors:0 dropped:0 overruns:0 frame:0
          TX packets:6 errors:0 dropped:0 overruns:0 carrier:0
          collisions:0 txqueuelen:1000 
          RX bytes:2251 (2.1 KiB)  TX bytes:492 (492.0 b)
\end{verbatim}

This shows that the system used the \emph{/etc/network/interfaces} file to
configure the device for you automatically when the device was found.

\subsection{Checking Routing}

Now, you should exit super user mode and run the following:

\begin{verbatim}
$ netstat -rn
\end{verbatim}

You should see a line in the output that looks as follows:

\begin{verbatim}
Kernel IP routing table
Destination     Gateway         Genmask         Flags   MSS Window  irtt Iface
192.168.129.0   0.0.0.0         255.255.255.0   U         0 0          0 usb0
\end{verbatim}

This shows that the system used the \emph{/etc/network/interfaces} file
to configure your routing tables so that the IP packets are routed to
the proper devices for the proper address. In this case it shows that
the 192.168.129 class C address family is routed out the usb0 device,
which is correct.

\subsection{Testing Connectivity}

Now, that everything has happened as expected, it is time to test
connectivity between the Zaurus and the Debian box. You can do this by
running the following command:

\begin{verbatim}
$ ping 192.168.129.201
\end{verbatim}

You should see output similar to the following:

\begin{verbatim}
PING 192.168.129.201 (192.168.129.201) 56(84) bytes of data.
64 bytes from 192.168.129.201: icmp_seq=1 ttl=255 time=3.26 ms
64 bytes from 192.168.129.201: icmp_seq=2 ttl=255 time=1.15 ms
64 bytes from 192.168.129.201: icmp_seq=3 ttl=255 time=1.33 ms
64 bytes from 192.168.129.201: icmp_seq=4 ttl=255 time=1.56 ms
64 bytes from 192.168.129.201: icmp_seq=5 ttl=255 time=1.75 ms
64 bytes from 192.168.129.201: icmp_seq=6 ttl=255 time=0.940 ms
\end{verbatim}

As you can see 6 pings were sent and 6 echos were received. You should
see similar output in the results of your ping. To stop the ping press
Ctrl-C in the terminal window you are running the ping in. The time just
shows the unit of time which it took to get an echo back from each ping
packet.

If you got output that looked similar to the above then you are good and you
are ready to use this connection to do what ever you want.

\section{Usage}

After the previous has been performed all should be good. All one should
have to do is leave the sync cradle plugged into the box at all times and
put the Zaurus in and power it on and off as needed for
synchronization. \emph{Note:} The Zaurus does not have to be in powered
off state when it is plugged in the cradle, that is just how I decided
to present it in this document for consistency.

% Here I include a version of the fdl.tex file that I have modified so it
% can be included directly into ones Latex article rather than into a book.
\include{fdl}

% End the document.
\end{document}
