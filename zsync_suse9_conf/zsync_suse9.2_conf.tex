%
% Copyright (c) 2004, 2005 Andrew De Ponte.
% Permission is granted to copy, distribute and/or modify this document
% under the terms of the GNU Free Documentation License, Version 1.2
% or any later version published by the Free Software Foundation;
% with the Invariant Sections being Front-Cover Texts, with the
% Front-Cover Texts being abstract. A copy of the license is included
% in the section entitled "GNU Free Documentation License".
%

%
% FileName:     zsync_suse9_conf.tex
% Author:       Andrew De Ponte
% E-Mail:       cyphactor@socal.rr.com
% AIM:          HUNNYnNUTTS
% Description:  This document is a LaTeX document containing the steps which
%               I took to configure a stock SuSE 9.0 Linux system to handle
%               the synchronization of my Zaurus SL-5600.
%

% Select the document class
\documentclass{article}

\usepackage{fullpage}

% Set the document information
\title{Zaurus SuSE 9.2 Cradle Synchronization Configuration}
\author{Andrew De Ponte}
\date{\today}

% Begin the document
\begin{document}

% Now that i have set the information, I want to create the title of the
% document.
\maketitle

% Here I place a page break so that the abstract has its own space.
\newpage

% Here I put the abstract of the document so the readers can decide if this
% is the correct document to read.
\begin{abstract}
This document outlines the means used to configure a SuSE 9.2 system
to handle synchronization of a Zaurus using the cradle and the synchronization
software provided by http://zsrep.sourceforge.net.
\end{abstract}

% Here I place a page break so that the table of contents starts on its own
% page.
\newpage

% Here I place the copyright notice and license information in which we
% state that this document falls under the GNU FDL or GNU Free Documentation
% License.
\noindent Copyright \copyright 2004, 2005, 2006 Andrew De Ponte.\\
Permission is granted to copy, distribute and/or modify this document under
the terms of the GNU Free Documentation License, Version 1.2 or any later
version published by the Free Software Foundation; with the Invariant
Sections being Front-Cover Texts, with the Front-Cover  Texts being
abstract. A copy of the license is included in the section entitled "GNU
Free Documentation License".

% Here I place a page break so that the license information does not get
% combined with the table of contents.
\newpage

% Place the table of contents of the document here.
\tableofcontents

% Here I place a page break so that the table of contents exists in its own
% space and does not fall into the actual document space.
\newpage

% Here should be the actual content of the document.
\section{Disable Firewall}

The first thing which needs to be done to configure ones SuSE 9.2 system for
Zaurus synchronization over the USB Cradle is disable the standard firewall
via \emph{YAST}. This can be done by running \emph{YAST}, going to the
\emph{Security and Users} section, and clicking on \emph{Firewall}. At this
point one will be presented with one of two possible screens. The first allows
for selection of external and internal interfaces. If you have
gotten this screen then your firewall is already disabled. The second contains
a radio button selection which allows the user to edit their
current firewall configuration or disable their current configuration. If you
have gotten this screen select the \emph{Stop Firewall and Remove from Boot
  Process} radio button, click the \emph{Next} button, and click the
\emph{Next} button again. At this point a window should pop up notifying you
that the firewall has been turned off. Simply click the \emph{OK} button and
close \emph{YAST}.

Firewalls are a very nice security tool and the fact that they
can be configured to allow for this setup to work. It is much too broad of a
subject to be covered in this document. Hence, my suggestion would be to read
up on IP-tables and learn how to manually configure a firewall in
Linux. Disabling the firewall in this case eliminates a large number of
variables that could create problems in getting this setup working.

\section{Required Driver}

The stock SuSE 9.2 kernel has built-in driver support for
the Zaurus hence you need not do anything.  The driver (kernel module) which
is used for the Zaurus is the \emph{usbnet} driver.

\section{Network Configuration}

The Zaurus driver (\emph{usbnet}) is actually a driver which allows
a USB device to act as a network device. The power of this is that it allows
applications written for standard network protocols to work over the USB
connection between the Zaurus and the computer. Since this driver basically
creates a network device, one has to setup a configuration file so that the
network device is automatically brought up and configured by the hotplug
system, one of it's other specialties.  This driver (\emph{usbnet}) creates a
USB Ethernet device, normally named (\emph{usb0}).  This device needs to
have the address of \emph{192.168.129.1}, a class C network address, due to
the fact that the address of the Zaurus is \emph{192.168.129.201}, a
class C network address.

    \subsection{sysconfig}

    A common method for configuring network devices is to use device named
    files in \emph{/etc/sysconfig/network/} with specific options set for
    that specific device. The good thing about this method is that a program
    called \emph{ifup} (interface up) supports it. One might say, well who
    cares if \emph{ifup} supports it? Well, the hotplug system as it is
    setup supports the \emph{ifup} program. One would want the network
    device created by the Zaurus driver to be configured
    automatically! To achieve this one should create a file named
    \emph{ifcfg-usb0} in \emph{/etc/sysconfig/network/}. The file should
    contain the following:

    \begin{verbatim}
    BOOTPROTO='static'
    BROADCAST='192.168.129.255'
    IPADDR='192.168.129.1'
    MTU=''
    NETMASK='255.255.255.0'
    NETWORK='192.168.129.0'
    REMOTE_IPADDR=''
    STARTMODE='hotplug'
    UNIQUE=''
    \end{verbatim}

    \subsection{Zaurus PC Link}

    On the Zaurus end of things, the Zaurus must have its IP address set as
    well. This is done on Sharp ROMs by using their \emph{PC Link}
    application. Go to the \emph{PC Link} application on your Zaurus. It can
    be found on your Zaurus under the \emph{Settings} tab. Once in the
    \emph{PC Link} application, you want to make sure the Zaurus hostname is
    set to ``zaurus'' (without the quotes) and that the USB IP address of the
    Zaurus is set to ``192.168.129.201'' (without the quotes). You also want
    to make sure that the proper connection type is selected. The \emph{USB -
    TCP/IP (advanced)} option, should be the connection option that is
    selected in the drop down list for the connection type. If it is
    \emph{NOT} then select it. After, the above have been performed tap the
    \emph{OK} button to apply your changes.

\section{Zaurus and Desktop Meet}

It is now time for the Zaurus and the SuSE 9.2 Desktop to meet.

\subsection{Watching the Log File}

Now, one will want to know that everything is working properly. Hence, they
would want to see the log output of what ever they are doing. In this case one
wants to see the log output of the kernel, kernel modules, and hotplug
system. To do so one would run the following command in a terminal as super
user:

\begin{verbatim}
# tail -f /var/log/messages
\end{verbatim}

Keep this terminal open and visible throughout the following process for it
will display changes made to log file as things are performed.

\subsection{Plugging in the Cradle}

At this stage one should plug in their \emph{empty} Zaurus cradle into a
working USB port on their SuSE 9.2 box. Note: You will not see anything happen
in the \emph{/var/log/messages} tail yet because the cradle isn't the device,
the cradle is basically just a spiffy wire that holds the
Zaurus rather than just connecting the Zaurus.

\subsection{Plugging in the Zaurus}

At this stage one should take their Zaurus, \emph{in powered off state}, and
place it fully into the cradle. Note: You still won't see anything in the
\emph{/var/log/messages} tail yet because the Zaurus is off.

\subsection{Turning on the Zaurus}

Turn on your Zaurus, leaving it in the cradle. You should see in the
\emph{/var/log/messages} tail something similar to the following:

\begin{verbatim}
Jan 24 09:42:57 waddler kernel: usb 2-1: new full speed USB device using address 2
Jan 24 09:42:57 waddler kernel: usb 2-1: Product: SL-5600
Jan 24 09:42:57 waddler kernel: usb 2-1: Manufacturer: Sharp
Jan 24 09:43:04 waddler kernel: usb0: register usbnet at usb-0000:00:11.3-1, Sharp Zaurus, PXA-2xx based
Jan 24 09:43:04 waddler kernel: usbcore: registered new driver usbnet
Jan 24 09:43:25 waddler kernel: usb0: no IPv6 routers present
\end{verbatim}

The first line is the usb kernel portion recognizing the Zaurus as full speed
USB device and identifying what address it is using. The second and third
lines are the usb kernel portion outputting its Product and Manufacturer
information. The fourth line and fifth line are very important. They are
showing that the \emph{usb0} device is being registered to the \emph{usbnet}
kernel module. This is basically showing that the hotplug system detected your
Zaurus and then loaded the usbnet driver for it. The fifth line is showing
that the usbcore kernel module recognized and registered the usbnet driver.
The sixth line is just some output complaining because IPv6 is not setup for
the \emph{usb0} device.

SuSE 9.2 along with SuSE 9.1 are known to display a pop up window notifying
the user of new hardware. It seems that SuSE detects the Zaurus device as a
modem and prompts the user to configure the device as a modem. Simply accept
configuration, which will take you into YAST. Once, in YAST the window shows a
list of detected modems and configured modems. All you have to do is click the
\emph{Finish} button leaving the Zaurus as a detected, however not configured
modem. This should stop SuSE from prompting you for configuration anymore.

\subsection{Checking Configuration}

At this point you should now run the following as super user to check if the
driver was loaded properly and the device configured properly:

\begin{verbatim}
# /sbin/ifconfig usb0
\end{verbatim}

You should see the following:

\begin{verbatim}
usb0      Link encap:Ethernet  HWaddr 5E:E0:14:AE:96:16
          inet addr:192.168.129.1  Bcast:192.168.129.255  Mask:255.255.255.0
          inet6 addr: fe80::5ce0:14ff:feae:9616/64 Scope:Link
          UP BROADCAST RUNNING MULTICAST  MTU:1500  Metric:1
          RX packets:0 errors:0 dropped:0 overruns:0 frame:0
          TX packets:6 errors:0 dropped:0 overruns:0 carrier:0
          collisions:0 txqueuelen:1000
          RX bytes:0 (0.0 b)  TX bytes:484 (484.0 b)
\end{verbatim}

This shows that the hotplug system use the
\emph{/etc/sysconfig/network/ifcfg-usb0} file to configure the device for you
automatically when the device was found.

\subsection{Checking Routing}

Now, you should exit super user mode and run the following:

\begin{verbatim}
$ netstat -rn
\end{verbatim}

You should see a line in the output that looks as follows:

\begin{verbatim}
Kernel IP routing table
Destination     Gateway         Genmask         Flags   MSS Window  irtt Iface
192.168.129.0   0.0.0.0         255.255.255.0   U         0 0          0 usb0
\end{verbatim}

This shows that the hotplug system used the
\emph{/etc/sysconfig/network/ifcfg-usb0} file to configure your routing tables
so that the IP packets are routed to the proper devices for the proper
address. In this case it shows that the 192.168.129 class C address family is
routed out the usb0 device, which is correct.

\subsection{Testing Connectivity}

Now, that everything has happened as expected, it is time to test connectivity
between the Zaurus and the SuSE 9.2 box. You can do this by running the
following command:

\begin{verbatim}
$ ping 192.168.129.201
\end{verbatim}

You should see output similar to the following:

\begin{verbatim}
PING 192.168.129.201 (192.168.129.201) 56(84) bytes of data.
64 bytes from 192.168.129.201: icmp_seq=1 ttl=255 time=3.26 ms
64 bytes from 192.168.129.201: icmp_seq=2 ttl=255 time=1.15 ms
64 bytes from 192.168.129.201: icmp_seq=3 ttl=255 time=1.33 ms
64 bytes from 192.168.129.201: icmp_seq=4 ttl=255 time=1.56 ms
64 bytes from 192.168.129.201: icmp_seq=5 ttl=255 time=1.75 ms
64 bytes from 192.168.129.201: icmp_seq=6 ttl=255 time=0.940 ms
\end{verbatim}

As you can see 6 pings were sent and 6 echos were received. You should see
similar output in the results of your ping. To stop the ping press Ctrl-C in
the terminal window you are running
the ping in. The time just shows the unit of time which it took to get an echo
back from each ping packet.

If you got output that looked similar to the above then you are good and you
are ready to use this connection to do what ever you want.

\section{Usage}

After the previous has been performed all should be good. All one should
have to do is leave the sync cradle plugged into the box at all times and
put the Zaurus in and power it on and off as needed for
synchronization.

% Here I include a version of the fdl.tex file that I have modified so it
% can be included directly into ones Latex article rather than into a book.
\include{fdl}

% End the document.
\end{document}
