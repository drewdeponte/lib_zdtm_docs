%
% Copyright (c) 2003, 2004, 2005 Andrew De Ponte.
% Permission is granted to copy, distribute and/or modify this document
% under the terms of the GNU Free Documentation License, Version 1.2
% or any later version published by the Free Software Foundation;
% with the Invariant Sections being Front-Cover Texts, with the
% Front-Cover Texts being abstract. A copy of the license is included
% in the section entitled "GNU Free Documentation License".
%

%
% FileName:     pao.tex
% Author:       Andrew De Ponte
% E-Mail:       cyphactor@socal.rr.com
% AIM:          HUNNYnNUTTS
% Description:  This document is a LaTeX document containing the
%               assumptions and observations about the synchronization
%               protocol used by Sharps Zaurus SL-5600 v1.0 ROM and
%               Sharps Zaurus SL-5500 v3.10 ROM, in hopes to produce
%               a document that will contain enough information about
%               the protocol that it will allow people to write software
%               to interact with the Zaurus via this protocol.
%

\newcommand{\synctypes}{
  \begin{enumerate}
  \item To-Do 0x06.
  \item Calendar 0x01.
  \item Address Book 0x07.
  \end{enumerate}
}

% Here I discuss the assumptions and observations of the
% protocol used for synchronization between a computer and
% the Zaurus SL-5600.
\part{Protocol Assumptions and Observations}

    % Here I describe the layer of the protocol stack at which
    % this protocol sits.
    \section{Protocol Layer}

    The protocol used for Synchronization between Sharps Microsoft Windows
    version of Qtopia Desktop and the Zaurus SL-5600 is what could be
    considered an application layer protocol. It sits directly on top of
    TCP (Transmission Control Protocol).

    % Here I discuss the the idea of the protocol being connection
    % based and its advantages etc.
    \section{Connection Based}

    The protocol at hand is \emph{connection-based} rather than connection-less
    like a protocol would be if it sat on top of UDP (User Datagram Protocol),
    or even the raw sockets layer. The fact that it is \emph{connection-based}
    provides it with the advantage of having data assurance handled by
    the TCP layer rather than having to handle it at the software layer.

    % Here I discuss the findings about the connection initiation for
    % this protocol.
    \section{Connection Initiation}

    This protocol handles two positions from which connections can be
    initiated.

        \subsection{Zaurus}

        One position is when the synchronization is initiated by the Zaurus.
        In this case the Zaurus creates a TCP connection to the Desktop on
        port 4245.

        \subsection{Desktop}

        The second position is when the synchronization is initiated by the
        Desktop. In this case the Desktop creates a TCP connection to the
        Zaurus on port 4244. The Desktop then sends a RAY message over this
        connection to the Zaurus. The Zaurus then responds to the previous
        mentioned data by creating a TCP connection to the Desktop as it
        does in the first position.

    % Here I discuss messages that are commonly seen and what they are
    % used for.
    \section{Common Messages}

    A number of messages are common within this protocol. Common messages
    are messages that are byte for byte the same despite their origin.

        \subsection{Request Message}

        The \emph{Request Message} is a message that is used to notify
        the member at the opposing side of communication that the sending
        side is ready to receive a message. The message is as follows.

        \begin{verbatim}
        ascii:  NUL  NUL  NUL  NUL  NUL       ENQ
          dec:    0    0    0    0    0  150    5 
          hex: 0x00 0x00 0x00 0x00 0x00 0x96 0x05
        \end{verbatim}

        \subsection{Acknowledgment Message}

        The \emph{Acknowledgment Message} is a message that is used to notify
        the member at the opposing side of communication that the sending
        side received the previously sent message. The message is as follows.

        \begin{verbatim}
        ascii:  NUL  NUL  NUL  NUL  NUL       ACK
          dec:    0    0    0    0    0  150    6 
          hex: 0x00 0x00 0x00 0x00 0x00 0x96 0x06
        \end{verbatim}

        \subsection{Abort Message}

        The \emph{Abort Message} is a message that is used to notify the
        member at the opposing side of communication that the sending
        side is aborting the synchronization. The message is as follows.

        \begin{verbatim}
        ascii:   NUL  NUL  NUL  NUL  NUL       CAN
          dec:     0    0    0    0    0  150   24
          hex:  0x00 0x00 0x00 0x00 0x00 0x96 0x18
        \end{verbatim}

    % Here I discuss the message format (the format used for non common
    % messages).
    \section{Message Format}

    Both Qtopia Desktop as well as the Zaurus use a pretty basic message
    format. It consists of four parts, a \emph{message header}, a
    \emph{message body size}, a \emph{message body},
    and a \emph{message checksum}. Note: \emph{This message format only
    applies to non common messages.}

    \begin{verbatim}
    [ Header (13 bytes) ] [ Body Size (2 bytes) ] [ Body (X bytes) ] [ Checksum (2 bytes) ]
    \end{verbatim}

        \subsection{Message Header}

        The \emph{message header} consists of the first 13 bytes of the
        message. The \emph{message header} is shared by all messages
        originating from the same object. There are two types of
        \emph{message headers}, the Zaurus \emph{message header} and
        the Qtopia Desktop \emph{message header}.

            \subsubsection{Zaurus Message Header}

            The Zaurus message header consists of the following
            13 static bytes.

            \begin{verbatim}
            ascii:  NUL  NUL  NUL  NUL  NUL       SOH  SOH  NUL
              dec:    0    0    0    0    0  150    1    1    0  255  255  255  255
              hex: 0x00 0x00 0x00 0x00 0x00 0x96 0x01 0x01 0x00 0xff 0xff 0xff 0xff
            \end{verbatim}

            \subsubsection{Qtopia Desktop Message Header}

            \label{qtopia:msgheader}
            The Qtopia Desktop message header
            consists of three sections, 1 section of 9 static bytes,
            followed by 2 variable bytes representing the
            \emph{message content size}, followed by 2 static bytes.
            This looks as follows:
            
            \begin{verbatim}
            ascii:  NUL  NUL  NUL  NUL  NUL       SOH  SOH   FF            NUL  NUL
              dec:    0    0    0    0    0  150    1    1   12 [  ] [  ]    0    0
              hex: 0x00 0x00 0x00 0x00 0x00 0x96 0x01 0x01 0x0c [  ] [  ] 0x00 0x00
            \end{verbatim}

            \emph{Note:} The [  ] represents a variable byte. In the above
            case one of the 2 bytes from the \emph{message content size}.

                \paragraph{Message Content Size}
                The \emph{message content size}
                is represented by two bytes. It is the
                \emph{message body size} minus
                \emph{the message type size (3 bytes)}.
                \emph{Note:} See the \emph{section \ref{msgtype}} for info
                on \emph{message type}.
                The right most byte is basically a counter for 256s. If the
                \emph{message content size} is larger than 255 the
                right most byte should represent the proper multiplier while
                the left most byte contains the remainder (less than 256).

                \begin{verbatim}
                [ Remainder (lsb) ] [ 256s Counter (msb) ]
                \end{verbatim}
                
                This layout of bytes is know to be
                the little endian byte order. Little endian is NOT the
                standard byte order for networking. So in this case the
                \emph{message content size} would be
                represented in C/C++ as an \emph{unsigned short integer}.
                If the host box (Desktop) is a big endian box then it would
                have to be converted from little endian to big endian to get
                the proper value.
            
        \subsection{Message Body Size}
        
        The \emph{message body size} is the size of the \emph{message body}
        in bytes. It is represented by two bytes. These two bytes are NOT
        represented in the standard byte order for networking (big endian).
        They are represented in little endian. This means that one should
        swap the byte order if their box is a big endian box.
        For more information on little endian
        refer to section \ref{qtopia:msgheader}.

        \begin{verbatim}
        [ Remainder (lsb) ] [ 256s Counter (msb) ]
        \end{verbatim}

        \subsection{Message Body}

        The message body consists of two parts, the \emph{message type},
        and the \emph{message content}.

        \begin{verbatim}
        [ Message Type (3 bytes) ] [ Message Content (X bytes) ]
        \end{verbatim}

            \subsubsection{Message Type}
            \label{msgtype}
            The \emph{message type}
            consists of 3 bytes which are of values that are displayable
            as alphabetic ASCII characters (A-Z). The \emph{message type}
            bytes are the first three bytes in the \emph{message body}.
            For information on the known \emph{message types} please
            refer to sections \ref{qtopia:msgs} and \ref{zaurus:msgs}.

            \subsubsection{Message Content}
            The \emph{message content}
            consists of a variable number of bytes which directly
            follow the \emph{message type}. This is the data that goes
            along with the specified \emph{message type}.
            
        \subsection{Message Checksum}

        The \emph{message checksum} is the summation of each byte within
        the \emph{message body}. It is represented as two bytes in the
        little endian byte order. In practice the byte order should only be
        swapped if your box is a big endian box.
        For more information on little endian refer to section
        \ref{qtopia:msgheader}.

        \begin{verbatim}
        [ Remainder (lsb) ] [ 256s Counter (msb) ]
        \end{verbatim}

    \section{Qtopia Desktop Messages}

    \label{qtopia:msgs}
    Within this protocol there exist a number of messages which only
    originate from Qtopia Desktop. I consider these Qtopia Desktop
    originated messages \emph{Qtopia Desktop Messages}.

        \subsection{RAY}

            \subsubsection{Content}

            The \emph{RAY} message contains no content.

            \subsubsection{Purpose}

            The purpose of this message seems to be to get a response of a
            message of type \emph{AAY}. This is the first message sent from
            the Desktop synchronization software to the Zaurus after the
            Zaurus connects to the Desktop synchronization software. It seems
            that it is used for initiation of the synchronization.

        \subsection{RIG}

            \subsubsection{Content}

            The \emph{RIG} message contains no content.

            \subsubsection{Purpose}

            The purpose of this message seems to be to request general
            information from the Zaurus (Model, Language, Auth State, etc) due
            to the fact that it receives a \emph{AIG} message in response
            which contains such information.

        \subsection{RRL}

            \subsubsection{Content}

            The \emph{RRL} message contains the passcode that should be sent
            to the Zaurus for authentication along with the size of the
            passcode in bytes. The passcode size is represented by a single
            byte preceding the password itself. The format of this message is
            as follows:

            \begin{verbatim}
              [ Passcode Size (1 Byte) ] [ Passcode (X Bytes) ]
            \end{verbatim}

            \begin{enumerate}
            \item Passcode Size (1 Byte) - This 1 byte represents the size of
              the following passcode in bytes.

            \item Passcode (X Bytes) - This variable number of bytes should
              contain the passcode itself.
            \end{enumerate}
            
            \subsubsection{Purpose}

            This message is used to send the passcode to the Zaurus for
            authentication before the Zaurus will allow synchronization to
            happen. This message is only sent to the Zaurus if the
            authentication state found in the \emph{AIG} message represents
            that the passcode option is enabled on the Zaurus. Please refer to
            \ref{zmsg:aig} for more info.

            \subsubsection{Note}

            This message is only needed if the passcode option on the Zaurus
            is enabled.

            \subsubsection{Security Note}

            The passcode is passed in clear text. Hence, one should not
            synchronize over public networks (or rather any place someone
            could possibly sniff their traffic, such as a open wireless
            network). Synchronizing using the cradle should not be a problem
            unless your box is owned by someone else.

        \subsection{RMG}

            \subsubsection{Content}

            The \emph{RMG} message contains two bytes of data, one of which
            differs depending on the type of synchronization. The format of the
            message is as follows.

            \begin{verbatim}
              [ UK (1 Byte) ] [ SYNC TYPE (1 Byte) ]
            \end{verbatim}

            \begin{enumerate}
            \item UK (1 Byte) - The meaning of this one byte is unknown. It
              has consistently been seen with a value of 0x01 in hex, despite
              variables such as synchronization type, etc.

            \item SYNC TYPE (1 Byte) - This byte represents the type of
              synchronization that is occurring. The known values in hex are
              as follows.

              \synctypes

            \end{enumerate}
            
            \subsubsection{Purpose}

            The purpose of this message seems to be to request the
            synchronization log from the Zaurus or at least a signature of some
            sort of the synchronization log. I make this assumption based on
            the fact that this message receives a \emph{AMG} message in
            response, and the \emph{AMG} message contains either the log or a
            signature of it.

        \subsection{RMS}

            \subsubsection{Content}

            The \emph{RMS} contains 40 bytes of data. There have been two
            cases that have been seen. The first two bytes of data refer to
            the size of the log file content. Unlike most messages if the log
            content is less then 38 bytes of data it is padded with 0x00s to
            be a total of 38 bytes of data.
            
            \subsubsection{Purpose}

            This message is used when doing a \emph{full sync}, or rather
            \emph{slow sync}. It sets the log file record kept on the
            Zaurus. This message is different then most though. Before it can
            be used to set the log file record on the Zaurus it has to be used
            to signify that the log file record is going to be reset. This is
            done by sending an \emph{RMS} message containing all 0x00 bytes as
            its content. After sending such an \emph{RMS} message one should
            receive an \emph{ANG} message containing a value of 0x83.

            After it has been notified that it is going to reset the log
            record another \emph{RMS} message should be sent to the Zaurus
            containing the log record for the Zaurus to store.

        \subsection{RSS}

            \subsubsection{Content}

            The \emph{RSS} message contains three bytes, one of which is the
            the type of synchronization. The format is as follows:

            \begin{verbatim}
              [ UK (1 Byte) ] [ SYNC TYPE (1 Byte) ] [ UK (1 Byte) ]
            \end{verbatim}

            \begin{enumerate}
            \item UK (1 Byte) - The meaning of this one byte is unknown. It
              has consistently been seen with a value of 0x01 in hex, despite
              variables such as synchronization type, etc.

            \item SYNC TYPE (1 Byte) - This byte represents the type of
              synchronization that is occurring. The known values in hex are
              as follows.

              \synctypes

            \item UK (1 Byte) - The meaning of this one byte is unknown. It
              has consistently been seen with a value of 0x01 in hex, despite
              variables such as synchronization type, etc.

            \end{enumerate}            

            \subsubsection{Purpose}

            This message looks to actually reset the state of items in the
            database on the Zaurus so that they will be recognized as new
            items.

        \subsection{RTG}

            \subsubsection{Content}

            The \emph{RTG} message contains no content.
            
            \subsubsection{Purpose}

            This message looks to be a request for the time stamp of the last
            synchronization on the Zaurus due to the fact the the Zaurus
            responds with an \emph{ATG} message.

        \subsection{RTS}
        
            \subsubsection{Content}
            
            The \emph{RTS} contains the current time
            of the Desktop box when sending the RTS message.
            The content of this message includes the time stamp in the format
            YYYYMMDDhhmmss in which each byte represents the ASCII value of
            the appropriate digit.

            \subsubsection{Purpose}

            At this point it is assumed that the purpose of this message
            is to notify the Zaurus of the time at which it was synchronized
            so that its software can do some comparison as to which things
            need to be synchronized as well as store it as an anchor for use
            in the next synchronization.

        \subsection{RDI}

            \subsubsection{Content}

            The \emph{RDI} contains two bytes, one of which differs depending
            on the type of synchronization (the synchronization type
            identifier). The format of the message is as follows.

            \begin{verbatim}
              [ SYNC TYPE (1 Byte) ] [ UK (1 Byte) ]
            \end{verbatim}

            \begin{enumerate}
            \item SYNC TYPE (1 Byte) - This byte represents the type of
              synchronization that is occurring. The known values in hex are
              as follows.

              \synctypes

            \item UK (1 Byte) - The meaning of this one byte is unknown. It
              has consistently been seen with a value of 0x07 in hex, despite
              variables such as synchronization type, etc.
            \end{enumerate}
            
            \subsubsection{Purpose}

            The purpose of this message seems to be to get a response of
            message type \emph{ADI}. Due to the fact that the \emph{ADI}
            message returns the format for the synchronization type, I assume
            that the actual purpose of this message is to request the data
            format for a specific type of synchronization (To-Do, Calendar,
            Address Book).

            \subsubsection{Note}

            This message is usually used once and then the format is stored. If
            the synchronization data is cleared or there are no items in the
            synchronization the \emph{RDI} message is sent to obtain the
            format for the synchronization.

        \subsection{RSY}

            \subsubsection{Content}

            The \emph{RSY} contains two bytes, one of which differs depending
            on the type of synchronization (the synchronization type
            identifier). The format of the message is as follows.

            \begin{verbatim}
              [ SYNC TYPE (1 Byte) ] [ UK (1 Byte) ]
            \end{verbatim}

            \begin{enumerate}
            \item SYNC TYPE (1 Byte) - This byte represents the type of
              synchronization that is occurring. The known values in hex are
              as follows.

              \synctypes

            \item UK (1 Byte) - The meaning of this one byte is unknown. It
              has consistently been seen with a value of 0x07 in hex, despite
              variables such as synchronization type, etc.
            \end{enumerate}
            
            \subsubsection{Purpose}

            The purpose of this message seems to be to get a response of
            message type \emph{ASY}. Since the \emph{ASY} message returns a
            list of synchronization IDs associated with items that need to be
            synchronized (new, deleted, or modified). I naturally assume that
            this message is a request for a list of all the synchronization
            IDs of items in a specific type (To-Do, Calendar, Address Book)
            that need to be synchronized.

            \subsubsection{Note}

            This message is used right before the actual content being
            synchronized is sent over. Notice that the two bytes seem to match
            with the two bytes sent over in the \emph{RDI}
            message. Specifically the unknown byte.

        \subsection{RDR}

            \subsubsection{Content}

            The \emph{RDR} contains a synchronization ID and some other bytes
            that are unknown as of this point. Below is the format for the
            body of the message as I know it (LOFD = length of following data,
            UK = unknown).

            \begin{verbatim}
              [ SYNC TYPE (1 Byte) ] [ NUM SYNC IDs (2 Bytes) ]
              [ SYNC ID (4 Bytes) ]
            \end{verbatim}

            \begin{enumerate}
            \item SYNC TYPE (1 Byte) - This byte represents the type of
              synchronization that is occurring. The known values in hex are
              as follows.

              \synctypes

            \item NUM SYNC IDs (2 Bytes) - These two bytes I believe represent
              the number of synchronization IDs that follow. \emph{Note: I
              have only ever seen 1 synchronization ID follow.} Hence, the
              value of this data should be 1 when interpreted as a little
              endian unsigned integer.

            \item SYNC ID (4 Bytes) - These four bytes contain the
            synchronization ID of the item which this message is requesting
            the data for. It is a 4 byte integer in little endian byte order.
            \end{enumerate}
            
            \subsubsection{Purpose}

            The purpose of this message seems to be to get a response of
            message type \emph{ADR}. This message basically seems to be a
            request for data on a specific item given the items associated
            synchronization ID and the type of synchronization.

            \subsubsection{Note}

            A series of these messages are usually sent to the Zaurus to
            request the data for each of the synchronization IDs that was
            received in the a previous \emph{ASY} message.

        \subsection{RDW}

            \subsubsection{Variations}

            The \emph{RDW} message has 3 different variations that I have
            seen. The first variation is seen when items that exist on both
            the Zaurus and the Desktop synchronization software are modified
            on the Desktop synchronization software side. When the next
            synchronization is performed a variation of the \emph{RDW} message
            is sent to the Zaurus for each item that has been modified.

            The second and third variations are both used when an item has
            been added to the Desktop synchronization software but has not
            been synchronized with the Zaurus yet. The first of these
            variations is used to acquire a synchronization ID for the new
            item. Once the synchronization ID has been acquired the second
            variation is used to send the item specific data (Category, Notes,
            Desc, etc.) to the Zaurus.

            \subsubsection{Content}

            The first of the 3 variations, the modified item variation,
            contains both content that is generic to all types of
            synchronization (To-do, Calender, Address) as well as content that
            is specific to the type of synchronization. The content is
            described in the following format. (LOFD = length of following
            data, UK = unknown).

            Below is the format for the generic content of the \emph{RDW}
            message.

            \begin{verbatim}
              [ SYNC TYPE (1 Byte) ] [ UK (2 Bytes) ]
              [ Sync ID - SYID (4 Bytes) ]
              [ UK (16 Bytes) ]
            \end{verbatim}

            \begin{enumerate}
            \item SYNC TYPE (1 Byte) - Is a 1 byte identifier used to specify
              the type of synchronization. The known values are as 
              follows in hex.

              \synctypes

            \item UK (2 Bytes) - These two bytes have an unknown meaning. They
              have consistently been seen to have the values 0x01 0x00 in
              hex. This value holds despite the type of synchronization. It
              may represent the number of items to be modified that are
              contained within the message. However, I have done some testing
              for this and have run into getting failure responses from the
              Zaurus when trying to send multiple within one message. Beyond
              that I have only seen a message modify one item at a time in
              working synchronizations.

            \item Sync ID - SYID (4 Bytes) - Is 4 bytes of data representing
              the synchronization ID associated with the item that this
              message should modify. It is a 4 byte unsigned integer in little
              endian byte order.

            \item UK (16 Bytes) - These 16 bytes have an unknown
              meaning. Each of these 16 bytes has consistently been seen to
              have  a value of 0xff in hex. These values hold despite type of
              synchronization. At this point I am just assuming that these 16
              bytes are used as some sort of separation buffer.
            \end{enumerate}

            The following format is the synchronization type specific data
            format.

            Below is format for the To-do type of synchronization.

            \begin{verbatim}
              [ LOFD (4 Bytes) ] [ Category - CTGR (X Bytes) ]
              [ LOFD (4 Bytes) ] [ Start Date - ETDY (5 Bytes) ]
              [ LOFD (4 Bytes) ] [ Due Date - ITDY (5 Bytes) ]
              [ LOFD (4 Bytes) ] [ Completed Date - FNDY (5 Bytes) ]
              [ LOFD (4 Bytes) ] [ Progress Status - MARK (1 Byte) ]
              [ LOFD (4 Bytes) ] [ Priority - PRTY (1 Byte) ]
              [ LOFD (4 Bytes) ] [ Description - TITL (X Bytes) ]
              [ LOFD (4 Bytes) ] [ Notes - MEM1 (X Bytes) ]
            \end{verbatim}

            \begin{enumerate}
              \item Category - CTGR (X Bytes) - Is the category associated
              with the To-Do item. It is a byte array of the ASCII values that
              make up the Category.

              \item Start Date - ETDY (5 Bytes) - Is the time at which the
              To-Do item has as its start date. It is in the same format as the
              \emph{Card Created Date Time} found in the common part of the
              \emph{ADR} message. This is NULL if the date is not set.

              \item Due Date - ITDY (5 Bytes) - Is the time the To-Do item has
              as its due date. It is in the same format as the
              \emph{Card Created Date Time} found in the common part of the
              \emph{ADR} message. This is NULL if the date is not set.

              \item Completed Date - FNDY (5 Bytes) - Is the time the To-Do
              item has as its completed date. It is in the same format as the
              \emph{Card Created Date Time} found in the common part of the
              \emph{ADR} message. This is NULL if the date is not set.

              \item Progress Status - MARK (1 Byte) - Is a single byte that
              represents the progress status of the To-Do item. If the To-Do
              item is completed then it is a value of 0. If the To-Do item is
              NOT completed then it is a value of 1.

              \item Priority - PRTY (1 Byte) - Is a single byte that
              represents the priority associated with the To-Do item. The value
              can range from 1 to 5 with 1 being the highest priority and 5
              being the lowest priority.

              \item Description - TITL (X Bytes) - Is a UTF-8 string of the
              Description associated with the To-Do item. It is a byte array of
              the UTF-8 values that make up the Description.

              \item Notes - MEM1 (X Bytes) - Is a UTF-8 string of the Notes
              associated with the To-Do item. It is a byte array of the UTF-8
              values that make up the Description.
            \end{enumerate}

            The second of the 3 variations, the obtain new sync ID variation,
            is used to obtain a new sync ID for new items on the Desktop
            synchronization software side. It's content is represented by the
            following format.

            \begin{verbatim}
              [ SYNC TYPE (1 Byte) ] [ UK (2 Bytes) ]
              [ UK (4 Bytes) ]
              [ LOFD (4 Bytes) ] [ Attribute - ATTR (1 Byte) ]
            \end{verbatim}

            \begin{enumerate}
            \item SYNC TYPE (1 Byte) - Is a 1 byte identifier used to specify
              the type of synchronization. The known values are as 
              follows in hex.

              \synctypes

            \item UK (2 Bytes) - These two bytes have an unknown meaning. They
              have consistently been seen to have the values 0x01 0x00 in
              hex. This value holds despite the type of synchronization. It
              may represent the number of items to be modified that are
              contained within the message. However, I have done some testing
              for this and have run into getting failure responses from the
              Zaurus when trying to send multiple within one message. Beyond
              that I have only seen a message modify one item at a time in
              working synchronizations.

            \item UK (4 Bytes) - These four bytes have an unknown
              meaning. They have consistently been seen to have the values of
              0x00 in hex. The values hold despite synchronization type. My
              assumption at this point is that it might represent the
              Synchronization ID and given that it is a value of 0 tells the
              Zaurus that it is a request for a new item. Hence, a new
              synchronization ID.

            \item LOFD (4 Bytes) - These four bytes represent the length of
            the following data in bytes. In this case since the Attribute
            field is the following data and the Attribute field only takes up
            one byte. This should have a value of 1 as far as I know.

            \item Attribute - ATTR (1 Byte) - This one byte is an attribute
              identifier. I have only seen it with a value of 0x00 in hex.
            \end{enumerate}

            The third of the 3 variations, the new item variation, is used to
            send a new Desktop synchronization software item data to the
            Zaurus so that it can be added to the proper database on the
            Zaurus. It's content is composed of two sections. One section
            which is generic to the type of synchronization and one that is
            specific to the type of the synchronization. The generic sections
            content is represented by the following format. 

            \begin{verbatim}
              [ SYNC TYPE (1 Byte) ] [ UK (2 Bytes) ]
              [ UK (4 Bytes) ]
              [ LOFD (4 Bytes) ] [ Attribute - ATTR (1 Byte) ]
              [ LOFD (4 Bytes) ] [ Card Created Date Time - CTTM (5 Bytes) ]
              [ LOFD (4 Bytes) ] [ Card Mod Date Time - MDTM (5 Bytes) ]
              [ LOFD (4 Bytes) ] [ Sync ID - SYID (4 Bytes) ]
            \end{verbatim}

            \begin{enumerate}
            \item SYNC TYPE (1 Byte) - Is a 1 byte identifier used to specify
              the type of synchronization. The known values are as 
              follows in hex.

              \synctypes

            \item UK (2 Bytes) - These two bytes have an unknown meaning. They
              have consistently been seen to have the values 0x01 0x00 in
              hex. This value holds despite the type of synchronization. It
              may represent the number of items to be modified that are
              contained within the message. However, I have done some testing
              for this and have run into getting failure responses from the
              Zaurus when trying to send multiple within one message. Beyond
              that I have only seen a message modify one item at a time in
              working synchronizations.

            \item UK (4 Bytes) - These four bytes have an unknown
              meaning. They have consistently been seen to have the values of
              0x00 in hex. The values hold despite synchronization type. My
              assumption at this point is that it might represent the
              Synchronization ID and given that it is a value of 0 tells the
              Zaurus that it is a request for a new item. Hence, a new
              synchronization ID.

            \item LOFD (4 Bytes) - These for bytes represent the length of the
              following data in bytes. In this case they are representing the
              length of the Attribute data, which is only one byte. Hence,
              these for bytes when interpreted as an unsigned long integer in
              little endian byte order should have a value of 1.

            \item Attribute - ATTR (1 Byte) - This one byte is an attribute
              identifier. I have only seen it with a value of 0x00 in hex.

            \item LOFD (4 Bytes) - These four bytes represent the length of
              the following data in bytes. In this case they are representing
              the length of the Card Created Date Time field. Since this is a
              new item to the Zaurus we do not want to specify a Card Created
              Date Time field. Hence, the value of these four bytes should be
              0.

            \item Card Created Date Time - CTTM (5 Bytes) - If the previous
              LOFD contains a value of 5 then this represents the date and time
              at which the card was created. If the LOFD contains a value of 0,
              which it should in this case, then these 5 bytes do not exist and
              the format continues.

            \item LOFD (4 Bytes) - These four bytes represent the length of
              the following data in bytes. in this case they are representing
              the length of the Cord Mod Date Time field. Since this is a new
              item to the Zaurus we do not want to specify a Card Mod Date
              Time field. Hence, the value of these four bytes should be 0.

            \item Card Mod Date Time - MDTM (5 Bytes) - If the previous LOFD
              contains a value of 5 then this represents the date and time at
              which the card was last modified. If the LOFD contains a value
              of 0, which it should in this case, then these 5 bytes do not
              exist and the format continues.

            \item LOFD (4 Bytes) - These four bytes represent the length of
              the following data in bytes. In this case they are representing
              the length of the Sync ID field. Since the Sync ID field contains
              only one Sync ID, and a Sync ID is an unsigned long integer,
              this should have a value of 4.

            \item Sync ID - SYID (4 Bytes) - Is 4 bytes of data representing
              the synchronization ID associated with the item that this
              message should create. It is a 4 byte unsigned integer in little
              endian byte order.
            \end{enumerate}

            The following format is the synchronization type specific data
            format.

            Below is format for the To-do type of synchronization.

            \begin{verbatim}
              [ LOFD (4 Bytes) ] [ Category - CTGR (X Bytes) ]
              [ LOFD (4 Bytes) ] [ Start Date - ETDY (5 Bytes) ]
              [ LOFD (4 Bytes) ] [ Due Date - ITDY (5 Bytes) ]
              [ LOFD (4 Bytes) ] [ Completed Date - FNDY (5 Bytes) ]
              [ LOFD (4 Bytes) ] [ Progress Status - MARK (1 Byte) ]
              [ LOFD (4 Bytes) ] [ Priority - PRTY (1 Byte) ]
              [ LOFD (4 Bytes) ] [ Description - TITL (X Bytes) ]
              [ LOFD (4 Bytes) ] [ Notes - MEM1 (X Bytes) ]
            \end{verbatim}

            \begin{enumerate}
              \item Category - CTGR (X Bytes) - Is the category associated
              with the To-Do item. It is a byte array of the ASCII values that
              make up the Category.

              \item Start Date - ETDY (5 Bytes) - Is the time at which the
              To-Do item has as its start date. It is in the same format as the
              \emph{Card Created Date Time} found in the common part of the
              \emph{ADR} message. This is NULL if the date is not set.

              \item Due Date - ITDY (5 Bytes) - Is the time the To-Do item has
              as its due date. It is in the same format as the
              \emph{Card Created Date Time} found in the common part of the
              \emph{ADR} message. This is NULL if the date is not set.

              \item Completed Date - FNDY (5 Bytes) - Is the time the To-Do
              item has as its completed date. It is in the same format as the
              \emph{Card Created Date Time} found in the common part of the
              \emph{ADR} message. This is NULL if the date is not set.

              \item Progress Status - MARK (1 Byte) - Is a single byte that
              represents the progress status of the To-Do item. If the To-Do
              item is completed then it is a value of 0. If the To-Do item is
              NOT completed then it is a value of 1.

              \item Priority - PRTY (1 Byte) - Is a single byte that
              represents the priority associated with the To-Do item. The value
              can range from 1 to 5 with 1 being the highest priority and 5
              being the lowest priority.

              \item Description - TITL (X Bytes) - Is a UTF-8 string of the
              Description associated with the To-Do item. It is a byte array of
              the UTF-8 values that make up the Description.

              \item Notes - MEM1 (X Bytes) - Is a UTF-8 string of the Notes
              associated with the To-Do item. It is a byte array of the UTF-8
              values that make up the Description.
            \end{enumerate}

            \subsubsection{Purpose}

            This message is a request for data writing. It seems that the
            purpose of this message is to send data from the Desktop to the
            Zaurus to have it either create or modify an item. It is used to
            transfer data in the synchronization process. It receives an
            \emph{ADW} message in response.

            \subsubsection{Note}

            This message only contains data for one item. If a number of items
            are to be written a number of these messages are sent.

        \subsection{RDD}

            \subsubsection{Content}

            The \emph{RDD} message contains a synchronization type identifier
            and a synchronization ID. The specific format of the message is as
            follows.

            \begin{verbatim}
              [ SYNC TYPE (1 Byte) ] [ NUM SYNC IDs (2 Bytes) ]
              [ SYNC ID (4 Bytes) ]
            \end{verbatim}

            \begin{enumerate}
            \item SYNC TYPE (1 Byte) - Is a 1 byte identifier used to specify
              the type of synchronization. The known values are as 
              follows in hex.

              \synctypes

            \item NUM SYNC IDs (2 Bytes) - These two bytes seem to represent
              the number of synchronization IDs following. This is
              consistently a value of 1 though. This is due to the fact that I
              have only ever seen 1 synchronization ID follow. If multiple
              items need to be deleted then multiple \emph{RDD} messages are
              sent.

            \item SYNC ID (4 Bytes) - These four bytes contain the
              synchronization ID of the item this message is requesting be
              deleted from the Zaurus database.
            \end{enumerate}
            
            \subsubsection{Purpose}

            The purpose of this message is to provide a method for the desktop
            synchronization application to notify the Zaurus that an item
            needs to be deleted due to synchronization, and the specific one
            that does.

        \subsection{RDS}

            \subsubsection{Content}

            The \emph{RDS} message contains a synchronization type identifier
            followed by three bytes of data. The format of the message is as
            follows.

            \begin{verbatim}
              [ SYNC TYPE (1 Byte) ] [ UK (3 Bytes) ]
            \end{verbatim}

            \begin{enumerate}
            \item SYNC TYPE (1 Byte) - Is a 1 byte identifier used to specify
              the type of synchronization. The known values are as 
              follows in hex.

              \synctypes

            \item UK (3 Bytes) - These three bytes have unknown meaning. They
              have consistently been seen as having the following values.

              \begin{verbatim}
                hex: 0x07 0x00 0x00
              \end{verbatim}
            \end{enumerate}

            \subsubsection{Purpose}

            The purpose of this message seems to be to get a response of
            message type \emph{AEX}. This message is sent to state that the
            synchronization has been successfully performed.

            \subsubsection{Note}

            This first two bytes of data seem to be the same two
            bytes that appear in the \emph{RSY} and \emph{RDI}
            messages. Specifically the 0x07.

        \subsection{RQT}

            \subsubsection{Content}

            The \emph{RQT} contains three bytes of data which don't seem to
            change.

            \begin{verbatim}
               ascii: NUL NUL NUL
                 dec: 0   0   0
                 hex: 00  00  00
            \end{verbatim}
            
            \subsubsection{Purpose}

            The purpose of this message seems to be to get a response of
            message type \emph{AEX}. It also seems that this message initiates
            the connection termination on the Zaurus side.

        \subsection{RLR}

            \subsubsection{Content}

            The \emph{RLR} message contains one byte which represents the type
            of synchronization. The message format looks as follows:

            \begin{verbatim}
              [ SYNC TYPE (1 Byte) ]
            \end{verbatim}

            \begin{enumerate}
            \item SYNC TYPE (1 Byte) - Is a 1 byte identifier used to specify
              the type of synchronization. The known values are as 
              follows in hex.

              \synctypes
            \end{enumerate}
            
            \subsubsection{Purpose}

            The purpose of the \emph{RLR} message seems to be to make a
            request for a listing of \emph{all} of the synchronization IDs
            (DTM IDs) of items associated with the specified synchronization
            type. It receives a \emph{ALR} message in response.

        \subsection{RGE}

            \subsubsection{Content}

            The \emph{RGE} message contains 2 bytes which contain the length
            of the following string, followed by X number of bytes which make
            up the path to one of the DTM index or box files on the
            Zaurus. The format looks as follows.

            \begin{verbatim}
              [ LOFS (2 Bytes) ] [ FILE PATH (X Bytes) ]
            \end{verbatim}

            \begin{enumerate}
              \item LOFS (2 Bytes) - This should be the length of the
              following string in bytes. It is represented in little endian
              byte order.

              \item FILE PATH (X Bytes) - This is a string which represents a
              fully qualified path to a DTM index or box file. The file path
              seems to be one of the following:

              \begin{enumerate}
                \item /home/zaurus/Applications/dtm/SLTODO.BOX - This is the
                path to the To-Do Application's DTM box file.

                \item /home/zaurus/Applications/dtm/SLTODO.IDX - This is the
                path to the To-Do Application's DTM index file.

                \item /home/zaurus/Applications/dtm/SLFLER.BOX - This is the
                path to the Category DTM box file.

                \item /home/zaurus/Applications/dtm/SLFLER.IDX - This is the
                path to the Category DTM index file.
              \end{enumerate}
            \end{enumerate}

            \subsubsection{Purpose}

            The purpose of this message seems to be to request the content of
            either a DTM index file or a DTM box file. The content of either
            of those two types is returned in one or more \emph{AGE} messages.

            \subsubsection{Notes}

            This message type is only used in the initial synchronization to
            set the synchronization state. It is NOT used in synchronizations
            of items after the synchronization state has been set. This is
            used to take the current state of the Zaurus DTM database and
            basically move it over to the Desktop and move the Desktop
            database over to the Zaurus. After this initial synchronization
            has been performed the protocol only synchronizes the items which
            need to be synchronized.

    \section{Zaurus Messages}

    \label{zaurus:msgs}
    There also exist a number of messages in this protocol which originate
    from the Zaurus. I consider these Zaurus originated messages
    \emph{Zaurus Messages}.

        \subsection{AAY}

            \subsubsection{Content}

            The \emph{AAY} contains three bytes which as of now are as
            follows (they have not changed):

            \begin{verbatim}
              ascii: NUL DLE NUL
                dec: 0   16  0
                hex: 00  10  00
            \end{verbatim}

            \subsubsection{Purpose}

            At this point due to my lack of knowledge I know that it is a
            response to the \emph{RAY} message but I am not sure as to what
            the three bytes represent. Some more testing may produce some
            results.

        \subsection{AIG}
        \label{zmsg:aig}
            \subsubsection{Content}

            The \emph{AIG} contains numerous pieces of data that we know of as
            important data, such as the Zaurus model, language, authentication
            state, etc. Below is the format for the body of the message as I
            know it (LOFS = length of following string, UK = unknown).

            \begin{verbatim}
              [ LOFS (2 Bytes) ]
              [ Model STR (X Bytes) ] [ UK (5 Bytes) ]
              [ Language (2 Bytes) ] [ Auth State (1 Byte) ] [ UK (6 Bytes) ]
            \end{verbatim}

            \begin{enumerate}
              \item LOFS (2 Bytes) - These two bytes are the length of the
              following string in bytes represented in little endian byte
              order.

              \item Model STR (X Bytes) - This is a string identifying the
              model of the Zaurus. It should be LOFS bytes and in most cases
              should consist of \emph{SL-5600}, \emph{SL-5500}, or
              \emph{SL-6000}.

              \item UK (5 Bytes) - These five bytes are unknown as to what
              they represent. They have \emph{not} changed and appear as
              follows (in hex):

              \begin{verbatim}
                0xff 0xff 0x31 0xff 0x20
              \end{verbatim}

              \item Language (2 Bytes) - These two bytes represent the
              language of the Zaurus. In most cases this will be \emph{en} in
              ASCII.

              \item Auth State (1 Byte) - This byte as I know of represents
              at least the authentication state. It seems to differ based on
              the model of the Zaurus (SL-5500, SL-5600, SL-5500G). 

              The values seen and their associated descriptions using the
              Zaurus SL-5600 and SL-5500 are as follows:
              \begin{itemize}
                \item 0x08 - The password option is disabled on the Zaurus. 

                \item 0x0b - The password option is enabled on the Zaurus and
                this message is before authentication of the password has
                occurred.

                \item 0x09 - The password option is enabled on the Zaurus and
                this message is after authentication of the password has
                occurred.
              \end{itemize}

              The values seen and their associated descriptions using the
              Zaurus SL-5500G are as follows:
              \begin{itemize}
                \item 0x04 - The password option is disabled on the Zaurus.

                \item 0x07 - The password option is enabled on the Zaurus and
                this message is before authentication of the password has
                occurred.

                \item 0x05 - The password option is enabled on the Zaurus and
                this message is after authentication of the password has
                occurred.
              \end{itemize}

              \item UK (6 Bytes) - These 6 bytes are unknown. They have not
              changed through out the testing. The bytes values are as
              follows (in hex):

              \begin{verbatim}
                0xff 0x00 0x00 0x00 0x00 0x00
              \end{verbatim}

            \end{enumerate}

            \subsubsection{Purpose}

            This message is a response to the \emph{RIG} message, and its
            purpose is probably from what I have seen, simply, to provide the
            synchronization software with some general information
            about the Zaurus (Auth State, Model, Language, etc).

        \subsection{AMG}

            \subsubsection{Content}

            The \emph{AMG} message looks to contain a large number of NULL
            byte along with some other non NULL bytes. It seems to always
            start with SL. I believe these bytes to represent the
            synchronization log or at least a signature of the synchronization
            log.

            \subsubsection{Purpose}

            It is a response to the \emph{RMG} message. Given that it contains
            the synchronization log, I assume that the purpose is to is solely
            to send the synchronization log to the Desktop synchronization
            software. This is seemingly used to determine if a full
            (slow) sync is necessary. If the sync log is completely
            empty the checksum of the message should be 213. The third
            byte of the content of the message can be checked to
            determine if a specific type of sync log is empty. This is
            done by bitwise-anding the third byte of the message content
            with the appropriate sync types mask and checking for a
            resulting value of 0. If the resulting value is 0 then a
            full (slow) sync has already been performed for that sync
            type. The necessary masks for comparison are as follows:

            \begin{itemize}
            \item ToDo Mask \textbf{0x01}
            \item Calendar Mask \textbf{0x02}
            \item Address Book Mask \textbf{0x04}
            \end{itemize}

        \subsection{ATG}

            \subsubsection{Content}

            The \emph{ATG} message looks to contain the value of the
            \emph{RTS} message that was received by the Zaurus the last time a
            synchronization occurred. I assume that the Zaurus some how records
            the last time that it was synchronized. The content of this message
            includes the time stamp of the last \emph{RTS} message (the last
            time it was synchronized) in the format YYYYMMDDhhmmss in which
            each byte represents the ASCII value of the appropriate digit.

            \subsubsection{Purpose}

            I am unsure as to what the value of this message content is if it
            has not been synchronized before but that will eventually be
            figured out. It seems to be used just to tell the Desktop
            synchronization software the last time it was synchronized.

        \subsection{AEX}

            \subsubsection{Content}

            The \emph{AEX} message contains no content.

            \subsubsection{Purpose}

            I know that it is sent to the Desktop Synchronization software in
            response to the \emph{RRL}, \emph{RTS}, \emph{RDS}, and \emph{RQT}
            messages. It seems to be a statement of success.

        \subsection{ANG}

            \subsubsection{Content}

            The \emph{ANG} message contains only one byte of content. It has
            so far only been seen to be 0x83 in hex.

            \subsubsection{Purpose}

            The purpose of this message has not been determined yet. I know
            that it is sent to the Desktop Synchronization software in
            response to an \emph{RMS} message containing 40 bytes of
            0x00s. This \emph{RMS} message seems to be used to state that the
            Zaurus log record is going to be set.

        \subsection{ADI}

            \subsubsection{Content}

            The \emph{ADI} message contains what seems to be the format of the
            synchronization data in some aspect. Below is the format for the
            content of the message as I know it. (LOFS = length of following
            string, UK = unknown)

            \begin{verbatim}
              [ TOT NUM CARDS (4 Bytes) ]
              [ NUM PARAMS PER CARD (2 Bytes) ] [ UK (1 Byte) ]
              [ Y NUM DTM PARAMS (X Bytes) ]
              [ PARAM DATA IDS (Y Bytes) ]
              [ LOFS (2 Bytes) ] [ Desc Param 1 (X Bytes) ]
              [ LOFS (2 Bytes) ] [ Desc Param 2 (X Bytes) ]
              [ LOFS (2 Bytes) ] [ Desc Param . (X Bytes) ]
              [ LOFS (2 Bytes) ] [ Desc Param . (X Bytes) ]
              [ LOFS (2 Bytes) ] [ Desc Param Y (X Bytes) ]
            \end{verbatim}

            \begin{enumerate}
              \item TOT NUM CARDS (4 Bytes) - These 4 bytes of data
              represent the total number of cards existing in the box
              (synchronization type) specified in the \emph{RDI} message. This
              is represented as an unsigned long integer in little endian byte
              order.

              \item NUM PARAMS PER CARD (2 Bytes) - These 2 bytes of data
              represent the number of parameters associated with each card of
              this box (synchronization type) specified in the \emph{RDI}
              message. This is represented as an unsigned short integer in
              little endian byte order.

              \item UK (1 Byte) - This 1 byte is unknown as to what it
              represents. It seems to consistently have a value of 0x07 (hex).

              \item Y NUM DTM PARAMS (X Bytes) - This is a listing
              of the abbreviations of the parameters for the box
              (synchronization type) specified in the \emph{RDI} message. It
              should always consist of ATTRCTTMMDTMSYID, these are the 4 common
              parameters that all items contain. Following these should be the
              specific parameters for the type of synchronization. For example
              with a To-Do it should be CTGRETDYITDYFNDYMARKPRTYTITLMEM1. This
              represents all the To-Do  specific parameter abbreviations. Each
              abbreviation is 4 characters so the first would be CTGR which
              represents the Category. The second is ETDY which is the start
              date. This is X number of bytes long because it is totally
              dependent on Y (the number of DTM Parameters associated with the
              type of synchronization). In fact, it should be Y * 4 since
              there are 4 bytes (characters) per parameter.

              \item PARAM DATA IDS (Y Bytes) - This is an associated list of
              single byte identifiers which represent the  data types of the
              above listed parameters. This section is Y bytes long because Y
              represents the number of parameters associated with the
              synchronization, and each parameters has a single byte
              identifier associated with it. The identifiers are as follows
              (in hex):

              \begin{verbatim}
                Strings in SJIS: 0x01
                      Code Data: 0x01
                     Image data: 0x02
                    Hybrid text: 0x03
                           Time: 0x04
                          Hours: 0x05
                            Bit: 0x06
           Week/day of the week: 0x07
                 Unsigned short: 0x08
                           Word: 0x08
                   Binary block: 0x09
              Unsigned short GS: 0x0a
                        Word GS: 0x0a
                  Unsigned char: 0x0b
                         BARRAY: 0x0c
                     Color data: 0x0d
                           SXLS: 0x0e
                           SRTF: 0x0f
               Strings in UTF-8: 0x11
                  Unsigned Long: 0x12
              \end{verbatim}

              An example of the might look as follows:

              \begin{verbatim}
                0x06 0x04 0x04
              \end{verbatim}

              In the above example the 0x06 means that the ATTR parameter is
              of type 0x06 which is a Bit, and that parameters CTTM and MDTM
              are both of type 0x04 which is a Time entry.

              \item LOFS (2 Bytes) - This should be the length of the
              following string in little endian byte order.

              \item Desc Param Y (X Bytes) - This is a description of the
              parameter associated with it, based on position in the data,
              hence the Y (parameter identifier) associated with it. It should
              have an LOFS preceding it which specifies the length of it.

            \end{enumerate}

            \subsubsection{Purpose}

            This message is a response to the \emph{RDI} message, and its
            purpose is probably from what I have seen, simply, to provide the
            synchronization software with the format for the up and coming
            data.

            \subsubsection{Note}

            Since this is a response to an \emph{RDI} message and the
            \emph{RDI} message is only used when the Desktop Synchronization
            software needs to know the format, this message is not seen in
            every synchronization.

        \subsection{ASY}

            \subsubsection{Content}

            The \emph{ASY} message contains the synchronization IDs of all of
            the items that should be synchronized (new, deleted, and
            modified). Below is the format for the content of the message as I
            know it. (LOFD = length of following data, UK = unknown)

            \begin{verbatim}
              [ UK (1 Byte) ]
              [ Y NUM SYNC IDs (2 Bytes) ]
              [ SYNC ID 1 (4 Bytes) ]
              [ SYNC ID 2 (4 Bytse) ]
              [ SYNC ID . (4 Bytes) ]
              [ SYNC ID . (4 Bytes) ]
              [ SYNC ID Y (4 Bytes) ]
              [ UK (1 Byte) ]
              [ Y NUM SYNC IDs (2 Bytes) ]
              [ SYNC ID 1 (4 Bytes) ]
              [ SYNC ID 2 (4 Bytse) ]
              [ SYNC ID . (4 Bytes) ]
              [ SYNC ID . (4 Bytes) ]
              [ SYNC ID Y (4 Bytes) ]
              [ UK (1 Byte) ]
              [ Y NUM SYNC IDs (2 Bytes) ]
              [ SYNC ID 1 (4 Bytes) ]
              [ SYNC ID 2 (4 Bytse) ]
              [ SYNC ID . (4 Bytes) ]
              [ SYNC ID . (4 Bytes) ]
              [ SYNC ID Y (4 Bytes) ]
            \end{verbatim}

            \begin{enumerate}
              \item UK (1 Byte) - This one byte has been seen with a value of
              1 consistently. I have yet to see it change values. At this
              point I think it is just an identifier for the new item sync ID
              list.

              \item Y NUM SYNC IDs (2 Bytes) - These 2 bytes contain the
              number of synchronization IDs that follow. It is an unsigned
              short integer in little endian byte order.

              \item SYNC ID Y (4 Bytes) - This is one of any number of
              synchronization IDs that are listed in series. Each one is an
              unsigned long integer in little endian byte order.

              \item UK (1 Byte) - This one byte has been seen with a value of
              2 consistently. I have yet to see it change values. At this
              point I think it is just an identifier for the modified item
              sync id list.

              \item Y NUM SYNC IDs (2 Bytes) - These 2 bytes contain the
              number of synchronization IDs that follow. It is an unsigned
              short integer in little endian byte order.

              \item SYNC ID Y (4 Bytes) - This is one of any number of
              synchronization IDs that are listed in series. Each one is an
              unsigned long integer in little endian byte order.

              \item UK (1 Byte) - This one byte has been seen with a value of
              4 consistently. I have yet to see it change values. At this
              point I think it is just an identifier for the deleted item sync
              id list.

              \item Y NUM SYNC IDs (2 Bytes) - These 2 bytes contain the
              number of synchronization IDs that follow. It is an unsigned
              short integer in little endian byte order.

              \item SYNC ID Y (4 Bytes) - This is one of any number of
              synchronization IDs that are listed in series. Each one is an
              unsigned long integer in little endian byte order.
            \end{enumerate}

            \subsubsection{Purpose}

            This message is a response to the \emph{RSY} message. The purpose
            of it is to provide the desktop synchronization software with a
            lists of synchronization IDs associated with the items that need to
            be synchronized (new, deleted, or modified).

        \subsection{ADR}

            \subsubsection{Content}

            The \emph{ADR} message contains the data of an item that should be
            synchronized. Below is the format for the body of the message as I
            know it. (LOFD = length of following data, UK = unknown)

            The format layout has two parts, a common portion and a
            synchronization specific part. The common portion is a portion
            that is shared among all synchronizations while the
            synchronization specific part is specific to the type of
            synchronization. The synchronization specific part follows the
            common part in the message.

            The common part of the format looks as follows.

            \begin{verbatim}
              [ UK (2 Bytes) ]
              [ NUM PARAMS (2 Bytes) ]
              [ LOFD (4 Bytes) ] [ Attribute - ATTR (1 Byte) ]
              [ LOFD (4 Bytes) ] [ Card Created Date Time - CTTM (5 Bytes) ]
              [ LOFD (4 Bytes) ] [ Card Mod Date Time - MDTM (5 Bytes) ]
              [ LOFD (4 Bytes) ] [ Sync ID - SYID (4 Bytes) ]
            \end{verbatim}

            \begin{enumerate}

              \item UK (2 Bytes) - The meaning of these 2 bytes are unknown.
              However, they do seem to consistently be the following 2 values
              (in hex).

              \begin{verbatim}
                hex: 0x01 0x00
              \end{verbatim}

              The only thing I can think of, is that these two bytes may
              represent the number of items that follow. Although, I have only
              ever seen the data of 1 item be sent using this message. It
              would also mean that the request message for this would require
              a parameter to represent the number to request. Which it very
              well may. I need to do further testing to figure this out.

              \item NUM PARAMS (2 Bytes) - This seems to be the total number
              of parameters associated with the item. In the case of To-do this
              is always 0x0c 0x00. Rather 12 in decimal.

              \item LOFD (4 Bytes) - Is the length of the following data in
              bytes. It is 4 bytes in little endian byte order.

              \item Attribute - ATTR (1 Byte) - Is a single byte which
              represents the attribute. This seems to be 0 all of the time.

              \item Card Created Date Time - CTTM (5 Bytes) - Is the time at
              which the card was created. It is stored in 5 bytes in the
              following format (left to right) when looking at it in big
              endian byte order. \emph{Note:} The Zaurus sends it over in
              little endian byte order. Also note that in this representation
              the top item represents th left most field in the format. Also
              note, on a system that is little endian, when you perform
              bitwise operations (such as bit shifts) it assumes you are
              thinking of the content in big endian. Henc, it will swap the
              little endian bytes to big endian before it performs the bit
              wise operation (such as bit shifts) and then swap the byte order
              back to little endian.

              \begin{verbatim}
                unknown: 4 bits
                   year: 8 bits
                  month: 4 bits
              month day: 5 bits
                   hour: 5 bits
                    min: 6 bits
                   secs: 6 bits
                unknown: 2 bits
              \end{verbatim}

              \begin{enumerate}
              \item unknown - I am unsure of what these 4 bits represent.
              \item year - This is the number of years since 1900.
              \item month - The number of the month ranging from 1 to 12.
              \item month day - The number representing day of month, 1 to 31.
              \item hour - This is the hours in GMT time.
              \item min - The number of minutes after the hour, 0 to 59.
              \item secs - The number of seconds after the minute, 0 to 59 in
              most cases. It could go up to 61 to account for leap seconds if
              the Zaurus supports leap seconds.
              \item unknown - I am unsure of what these 2 bits represent.
              \end{enumerate}

              This format provides all that is needed to create a calendar
              time and date. Hence, I think the unknown bits are just
              garbage bits. There is a possibility that the 4 bits preceding
              the \emph{year} field could be part of the year. Without them
              though the year field has a max value of 255. This limits the
              year it represents to 2155. Considering that we are currently in
              2004 I do not think this limitation will be a problem. Beyond
              that the two bits have now changed but I have not narrowed down
              exactly what caused the change and have yet to find a
              significant reason to take note of the change.

              \item Card Mod Date Time - MDTM (5 Bytes) - Is the time at
              which the card was last modified. It is stored in 5 bytes in the
              same format as the \emph{Card Created Date Time}.

              \item Sync ID - SYID (4 Bytes) - This is a synchronization
              identifier. It is 4 bytes and is an unsigned long integer in
              little endian byte order.

            \end{enumerate}

            The synchronization specific portion for To-Do is as follows:

            \begin{verbatim}
              [ LOFD (4 Bytes) ] [ Category - CTGR (X Bytes) ]
              [ LOFD (4 Bytes) ] [ Start Date - ETDY (5 Bytes) ]
              [ LOFD (4 Bytes) ] [ Due Date - ITDY (5 Bytes) ]
              [ LOFD (4 Bytes) ] [ Completed Date - FNDY (5 Bytes) ]
              [ LOFD (4 Bytes) ] [ Progress Status - MARK (1 Byte) ]
              [ LOFD (4 Bytes) ] [ Priority - PRTY (1 Byte) ]
              [ LOFD (4 Bytes) ] [ Description - TITL (X Bytes) ]
              [ LOFD (4 Bytes) ] [ Notes - MEM1 (X Bytes) ]
            \end{verbatim}

            \begin{enumerate}
              \item Category - CTGR (X Bytes) - Is the category associated
              with the To-Do item. It is a byte array of the ASCII values that
              make up the Category.

              \item Start Date - ETDY (5 Bytes) - Is the time at which the
              To-Do item has as its start date. It is in the same format as the
              \emph{Card Created Date Time} found in the common part of this
              message. This is non existent if the date is not set and the
              preceding LOFD has a value of 0.

              \item Due Date - ITDY (5 Bytes) - Is the time the To-Do item has
              as its due date. It is in the same format as the
              \emph{Card Created Date Time} found in the common part of this
              message. This is non existent if the date is not set and the
              preceding LOFD has a value of 0.

              \item Completed Date - FNDY (5 Bytes) - Is the time the To-Do
              item has as its completed date. It is in the same format as the
              \emph{Card Created Date Time} found in the common part of this
              message. This is non existent if the date is not set and the
              preceding LOFD has a value of 0.

              \item Progress Status - MARK (1 Byte) - Is a single byte that
              represents the progress status of the To-Do item. If the To-Do
              item is completed then it is a value of 0. If the To-Do item is
              NOT completed then it is a value of 1.

              \item Priority - PRTY (1 Byte) - Is a single byte that
              represents the priority associated with the To-Do item. The value
              can range from 1 to 5 with 1 being the highest priority and 5
              being the lowest priority.

              \item Description - TITL (X Bytes) - Is a UTF-8 string of the
              Description associated with the To-Do item. It is a byte array of
              the UTF-8 values that make up the Description.

              \item Notes - MEM1 (X Bytes) - Is a UTF-8 string of the Notes
              associated with the To-Do item. It is a byte array of the UTF-8
              values that make up the Description.
            \end{enumerate}

            \subsubsection{Purpose}

            This message is a response to the \emph{RDR} message. It seems
            that the purpose of this message is to provide the data for the
            specified sync ID requested in the \emph{RDR} message.

            \subsubsection{Note}

            This message is sent as a response for every \emph{RDR} message
            that was sent. Hence, each \emph{ADR} message only contains the
            data for one item of type To-Do, Calendar, or Address Book.

        \subsection{ADW}

            \subsubsection{Content}

            The \emph{ADW} message contents are in accordance with the
            following format.

            \begin{verbatim}
              [ UK (4 Bytes) ] [ NUM IDs (2 Bytes) ] [ SYNC ID (4 Bytes) ]
            \end{verbatim}

            \begin{enumerate}
            \item UK (4 Bytes) - The meaning of these four bytes is
              unknown. They have consistently been seen with the value of 0x00
              in hex despite synchronization type or any other tested factors.
              \begin{verbatim}
                hex: 0x00 0x00 0x00 0x00
              \end{verbatim}

            \item NUM IDs (2 Bytes) - I believe these two bytes to represent
              the number of synchronization IDs following. I believe this
              despite the fact that I have only ever seen one synchronization
              ID follow. Beyond that, these two bytes have always been seen to
              have a value of 1 in little endian byte order.
              \begin{verbatim}
                hex: 0x01 0x00
              \end{verbatim}

            \item SYNC ID (4 Bytes) - These four bytes represent the
              synchronization ID of the item that was just created or modified
              by the previous \emph{RDW} message.
            \end{enumerate}
            
            \subsubsection{Purpose}

            The \emph{ADW} message is a message sent in response to an
            \emph{RDW} message. As far as I know it signifies success of the
            prior \emph{RDW} message.

        \subsection{ALR}

            \subsubsection{Content}
            
            The \emph{ALR} message contains two bytes representing the number
            of synchronization (DTM IDs) following, followed by some number of
            synchronization IDs (DTM IDs). If there are no items to be
            synchronized for the specified type the message only contains
            the 2 bytes representing the number of synchronization IDs
            (zero in this case). The format looks as follows:

            \begin{verbatim}
              [ NUM SYNC IDs (2 Bytes) ]
              [ SYNC ID 1 (4 Bytes) ] [ SYNC ID 2 (4 Bytes) ]
              [ SYNC ID 3 (4 Bytes) ] [ SYNC ID X (4 Bytes) ]
            \end{verbatim}

            \begin{enumerate}
              \item NUM SYNC IDs (2 Bytes) - This is the number of
              synchronization IDs that follow. It is an unsigned short integer
              in little endian byte order.

              \item SYNC ID X (4 Bytes) - Each of the SYNC IDs represents a
                synchronization ID of an item that is going to be sent over
                for the requested type of synchronization.
            \end{enumerate}

            \subsubsection{Purpose}

            The \emph{ALR} message's purpose is to respond to a \emph{RLR}
            message, containing all of the synchronization IDs (DTM IDs) of
            the items that are going to be sent over given the type of
            synchronization in the \emph{RLR} message.

        \subsection{AGE}

            \subsubsection{Content}

            The \emph{AGE} message is different then most messages. It has two
            different content layouts. One of layouts is for the \emph{first}
            \emph{AGE} message received after sending an \emph{RGE}. The other
            layout is for all \emph{AGE} message subsequent to the
            \emph{first} \emph{AGE} message.

            In the content layout for the \emph{first} \emph{AGE} message, the
            message contains four bytes which contain the \emph{total length}
            of file requested in the \emph{RGE} message, followed by X number
            of bytes which make up either the total content/partial content of
            either the DTM index file or the DTM box file requested. If the
            message contains only a portion of the file content then other
            \emph{AGE} messages should follow containing the rest of the file
            data. The format looks as follows.

            \begin{verbatim}
              [ LOFD (4 Bytes) ] [ FILE DATA (X Bytes) ]
            \end{verbatim}

            \begin{enumerate}
              \item LOFD (4 Bytes) - This is the total length of the file
              requested in the previous \emph{RGE} message, in bytes.

              \item FILE DATA (X Bytes) - This is the byte for byte data found
              in the file located at the path that was passed over in the RGE
              request message. This is as far as I have seen the content of
              one of the DTM index or box files.
            \end{enumerate}

            In the other content layout, the message contains X number of
            bytes which make up either the total content/partial content of
            either the DTM index file or the DTM box file requested. If the
            message contains only a portion of the file content then other
            \emph{AGE} messages should follow containing the rest of the file
            data. The format looks as follows.

            \begin{verbatim}
              [ FILE DATA (X Bytes) ]
            \end{verbatim}

            \begin{enumerate}
              \item FILE DATA (X Bytes) - This is the byte for byte data found
              in the file located at the path that was passed over in the RGE
              request message. This is as far as I have seen the content of
              one of the DTM index or box files.
            \end{enumerate}

            \subsubsection{Purpose}

            The purpose of this message seems to be to send the content of a
            DTM index file or a DTM box file. This is done in response to a
            \emph{RGE} message.

            \subsubsection{Notes}

            This message type is only used in the initial synchronization to
            set the synchronization state. This method of sync is know as
            \emph{Slow Sync} by the Sync-ML standard and as \emph{Full Sync} by
            the Qtopia Desktop software. It is \emph{not} used in
            synchronizations of items after the synchronization state has been
            set.

            \subsubsection{Security Note}
            The limitations of this request have \emph{not} been tested. It is
            possible that the access is \emph{not} limited and that this could
            be abused to obtain certain files from a person's Zaurus.

    \section{PIM Database Parameters}

    Below is a table of all the PIM database parameters organized by
    Application type. This table can be used as a guide for the datatypes of
    the parameters as well as a quick reference for parameter content and
    parameter IDs. The \emph{Application} column of the table contains the
    name of the application (synchronization type) that the listed parameters
    are used in. The \emph{Parameters} column of the table contains the
    parameter IDs of all the different types of parameters. The \emph{Type}
    column of the table specifies the datatype identifier for the data types
    which are used to transfer each parameter. The \emph{Content} column
    contains a basic description of what each parameter contains. The
    \emph{Remarks} column contains things that should be noted for specific
    parameters.

\small
\setlongtables
    \begin{longtable}[c]{|l|l|l|p{3cm}|p{5cm}|}
      \caption{PIM Database Parameters} \\
      \hline
      Applications & Parameters & Type & Content & Remarks \\
      \hline
      \endhead
      \hline
      \multicolumn{5}{|r|}{\small\slshape continued on next page} \\
      \hline
      \endfoot
      \hline
      \endlastfoot
%        Applications & Parameters & Type & Content & Remarks \\
        \hline
        \label{paramtable}
        Common & ATTR & DATA\_ID\_BIT & Attribute & \\
        & CTTM & DATA\_ID\_TIME & Card created date and time & UTC \\
        & MDTM & DATA\_ID\_TIME & Card modified date and time & UTC \\
        & SYID & DATA\_ID\_ULONG & 32bit sync ID & \\
        \hline

        Todo & CTGR & DATA\_ID\_BARRAY & Category & Category ID \\
        & ETDY & DATA\_ID\_TIME & Start Date & UTC, NULL if the date is not
        set (size in bytes = 0) \\
        & LTDY & DATA\_ID\_TIME & Due Date & UTC, NULL if the date is not set
        (size in bytes = 0) \\ 
        & FNDY & DATA\_ID\_TIME & Completed Date & UTC, NULL if the date is
        not set (size in bytes = 0) \\
        & MARK & DATA\_ID\_UCHAR & Progress Status & Completed = 0, Incomplete
        = 1 \\
        & PRTY & DATA\_ID\_UCHAR & Priority & 1(very high) - 5(very low) \\
        & TITL & DATA\_ID\_UTF8 & Description & \\
        & MEM1 & DATA\_ID\_UTF8 & Notes & \\
        \hline

        Calendar & CTGR & DATA\_ID\_BARRAY & Categar & Category ID \\
        & DSRP & DATA\_ID\_UTF8 & Description & \\
        & PLCE & DATA\_ID\_UTF8 & Location & \\
        & MEM1 & DATA\_ID\_UTF8 & Notes & \\
        & TIM1 & DATA\_ID\_TIME & Start Time & UTC, except All Day events are
        Zaurus local time \\
        & TIM2 & DATA\_ID\_TIME & End Time & UTC, except All Day events are
        Zaruus local time \\
        & ADAY & DATA\_ID\_UCHAR & Schedule Type & Normal = 0, All Day = 1 \\
        & ARON & DATA\_ID\_UCHAR & Alarm & On = 0, Off = 1 \\
        & ARMN & DATA\_ID\_WORD & Alarm time & As minutes \\
        & ARSD & DATA\_ID\_UCHAR & Alarm Setting & Silent = 0, Loud = 1 \\
        & RTYP & DATA\_ID\_UCHAR & Repeat type & Daily=0, Weekly=1,
        MonthlyDay=2, MonthlyDate=3, YearlyDate=4, Set 0xff in non-repeat \\
        & RFRQ & DATA\_ID\_WORD & Repeat period & \\
        & RPOS & DATA\_ID\_WORD & Repeat position & \\
        & RDYS & DATA\_ID\_UCHAR & Repeat date & MON=0x01, TUE=0x02, WED=0x04,
        THU=0x08, FRI=0x10, SAT=0x20, SUN=0x40 \\        
        & REND & DATA\_ID\_UCHAR & Repeat end date setting & Not Set = 0,
        Set = 1 \\
        & REDT & DATA\_ID\_TIME & Repeat end date & Not UTC \\
        & ALSD & DATA\_ID\_TIME & All Day start date & Zaurus Local Time, only
        date valid, time is garbage \\
        & ALED & DATA\_ID\_TIME & All Day end date & Zaurus Local Time, only
        date valid, time is garbage \\
        & MDAY & DATA\_ID\_UCHAR & Multiple days flag & Same Day = 0, Multiple
        Days = 1 \\
        \hline
        Addressbook & CTGR & DATA\_ID\_BARRAY & Category & Category ID \\
        & FULL & DATA\_ID\_UTF8 & Full name & Used in List view. \\
        & NAPR & DATA\_ID\_UTF8 & Full name pronunciation & \\
        & TITL & DATA\_ID\_UTF8 & Term of respect & \\
        & LNME & DATA\_ID\_UTF8 & Last name & \\
        & FNME & DATA\_ID\_UTF8 & First name & \\
        & MNME & DATA\_ID\_UTF8 & Middle name & \\
        & SUFX & DATA\_ID\_UTF8 & Suffix & \\
        & FLAS & DATA\_ID\_UTF8 & Alternative name & \\
        & LNPR & DATA\_ID\_UTF8 & Last name pronunciation & \\
        & FNPR & DATA\_ID\_UTF8 & First name pronunciation & \\
        & CPNY & DATA\_ID\_UTF8 & Company & \\
        & CPPR & DATA\_ID\_UTF8 & Company pronunciation & \\
        & SCTN & DATA\_ID\_UTF8 & Department & \\
        & PSTN & DATA\_ID\_UTF8 & Job title & \\
        & TEL2 & DATA\_ID\_UTF8 & Work phone & \\
        & FAX2 & DATA\_ID\_UTF8 & Work FAX & \\
        & CPS2 & DATA\_ID\_UTF8 & Work mobile & \\
        & BSTA & DATA\_ID\_UTF8 & Work state & \\
        & BCTY & DATA\_ID\_UTF8 & Work city & \\
        & BSTR & DATA\_ID\_UTF8 & Work street & \\
        & BZIP & DATA\_ID\_UTF8 & Work Zip & \\
        & BCTR & DATA\_ID\_UTF8 & Work country & \\
        & BWEB & DATA\_ID\_UTF8 & Work web page & \\
        & OFCE & DATA\_ID\_UTF8 & Office & \\
        & PRFS & DATA\_ID\_UTF8 & Profession & \\
        & ASST & DATA\_ID\_UTF8 & Assistant & \\
        & MNGR & DATA\_ID\_UTF8 & Manager & \\
        & BPGR & DATA\_ID\_UTF8 & Pager & \\
        & CPS1 & DATA\_ID\_UTF8 & Cellular & \\
        & TEL1 & DATA\_ID\_UTF8 & Home phone & \\
        & FAX1 & DATA\_ID\_UTF8 & Home FAX & \\
        & HSTA & DATA\_ID\_UTF8 & Home state & \\
        & HCTY & DATA\_ID\_UTF8 & Home city & \\
        & HSTR & DATA\_ID\_UTF8 & Home street & \\
        & HZIP & DATA\_ID\_UTF8 & Home Zip & \\
        & HCTR & DATA\_ID\_UTF8 & Home country & \\
        & HWEB & DATA\_ID\_UTF8 & Home web page & \\
        & DMAL & DATA\_ID\_UTF8 & Default email & First email in Emails \\
        & MAL1 & DATA\_ID\_UTF8 & Emails & Space seperated list \\
        & SPUS & DATA\_ID\_UTF8 & Spouse & \\
        & GNDR & DATA\_ID\_UTF8 & Gender & \\
        & BRTH & DATA\_ID\_UTF8 & Birthday & \\
        & ANIV & DATA\_ID\_UTF8 & Anniversary & \\
        & NCNM & DATA\_ID\_UTF8 & Nickname & \\
        & CLDR & DATA\_ID\_UTF8 & Children & \\
        & MEM1 & DATA\_ID\_UTF8 & Memo & \\
        & GRPS & DATA\_ID\_UTF8 & Group & \\
        \hline
        Memo & CLAS & DATA\_ID\_UTF8 & List view title & 20 chars from
        Description \\
        & MEM1 & DATA\_ID\_UTF8 & Description & \\
        & DATE & DATA\_ID\_TIME & Last modifide date & UTC \\
        \hline
        Email & MLID & DATA\_ID\_USHORT & Mail ID & \\
        & SIZE & DATA\_ID\_USHORT & Mail size & \\
        & TYPE & DATA\_ID\_UCHAR & Mail type & \\
        & RDCK & DATA\_ID\_UCHAR & Read/Unread & \\
        & DLCK & DATA\_ID\_UCHAR & Server download & \\
        & DECK & DATA\_ID\_UCHAR & Server delete & \\
        & SVID & DATA\_ID\_TEXT & Server ID & \\
        & SDCK & DATA\_ID\_UCHAR & Sent/Not sent yet & \\
        & MDCK & DATA\_ID\_UCHAR & Modified/Not modified & \\
        & DRFT & DATA\_ID\_UCHAR & Draft email & \\
        & ACNM & DATA\_ID\_TEXT & Account & \\
        & MLBX & DATA\_ID\_TEXT & Mail box & \\
        & RVTM & DATA\_ID\_TIME & Received date & \\
        & SDTM & DATA\_ID\_TIME & Sent date & \\
        & TMZN & DATA\_ID\_TEXT & Timezone & \\
        & FMNM & DATA\_ID\_TEXT & Sender & \\
        & FMAD & DATA\_ID\_TEXT & Sender's email address & \\
        & MLTO & DATA\_ID\_TEXT & Destination TO email address & \\
        & MLCC & DATA\_ID\_TEXT & Destination CC email address & \\
        & MLBC & DATA\_ID\_TEXT & Destination BCC email address & \\
        & MLRP & DATA\_ID\_TEXT & Replying email address & \\
        & SBJT & DATA\_ID\_TEXT & Subject & \\
        & BODY & DATA\_ID\_TEXT & Email body & \\
        & ENNM & DATA\_ID\_USHORT & Number of attachement(s) & \\
        & ENAT & DATA\_ID\_TEXT & Email attachement attribute & \\
        & ENTP & DATA\_ID\_TEXT & Email attachement file type & \\
        & ENFL & DATA\_ID\_TEXT & Attachement file name (full path) & \\
        & DMMY & DATA\_ID\_TEXT & Dummy & \\
        \hline
        Category & TITL & DATA\_ID\_UTF8 & Category title & \\
        & SCTG & DATA\_ID\_CATEGORY & Category ID & See TYPE fields to
        determine if it is a preset category or not \\
        & TYPE & DATA\_ID\_BIT & Attribute & bit7 (0x80) represents preset
        category \\
        & COLR & DATA\_ID\_COLOR & Book color & Not used \\
        \hline
    \end{longtable}

\normalsize
    \subsection{Datatype Descriptions}
    \label{type:desc}
    Below exist detailed descriptions of all the different datatype
    identifiers used in the \emph{PIM Database Parameters} table.

    \begin{enumerate}
    \item DATA\_ID\_BIT - This data type is stored within 1 byte. Thus far I
    have only seen it with values of 1 or 0, assumingly representing true and
    false.

    \item DATA\_ID\_TIME - This data type is stored within 5 bytes and is
    transmitted from the Zaurus in little endian byte order. This data type is
    used to represent date, time, or the combination of the two. Despite the
    fact that it is transmitted in little endian, it must be in big endian byte
    order to decompose it. In big endian byte order the format is as follows
    (left to right):

\begin{verbatim}
                unknown: 4 bits
                   year: 8 bits
                  month: 4 bits
              month day: 5 bits
                   hour: 5 bits
                    min: 6 bits
                   secs: 6 bits
                unknown: 2 bits
\end{verbatim}

    \item DATA\_ID\_ULONG - This data is stored in 4 bytes and is transmitted
    from the Zaurus in little endian byte order. This data type is used to
    represent the synchronization ID of an item and possibly other things.

    \item DATA\_ID\_BARRAY - This data is stored in X number of bytes. The X
    number of bytes are transmitted as an array of bytes. This data type is
    used to represent the category of an item. The byte array only consists of
    ASCII characters. This byte array represents the ID of the category an
    item is associated with.

    \item DATA\_ID\_UCHAR - This data is stored within 1 byte. It represents
    an unsigned char and can be represented as an unsigned char in C or
    C++. It is used to represent a number of things such as state of options,
    priority level, repeat type, etc.

    \item DATA\_ID\_UTF8 - This data is stored within X number of bytes. The X
    number of bytes are transmitted as UTF8 data. This data type used to
    transfer data for such things as Description, Location, Notes, etc.

    \item DATA\_ID\_WORD - This data is stored within 2 bytes and is
    transmitted from the Zaurus in little endian byte order. This data type is
    used to represent alarm time (in minutes), repeat period, repeat position,
    etc.

    \end{enumerate}

    \subsection{Parameter Exceptions}

    The information provided in Table \ref{paramtable} is quite extensive,
    however it does not cover some specific data handling specifications due
    to the limited remarks space. Such needed specifications are duly place in
    this section to allow for the depth of description needed.

    \subsubsection{Calendar Parameter Exceptions}

    \begin{enumerate}
      \item TIM1, TIM2 - The \emph{TIM1} and \emph{TIM2} parameters are used
      to represent the Start Date and Time and End Date and Time of an
      event. Normally, when an event is \emph{NOT} set as an \emph{All Day}
      event \emph{TIM1} and \emph{TIM2} contain the date and time in
      UTC. However, if the event is marked as an \emph{All Day} event
      \emph{TIM1} and \emph{TIM2} contain the date and time in the Zaurus
      local time. This limits how \emph{All Day} events can be mapped when
      synchronizing. For the desktop synchronization software to be able to
      properly convert from the Zaurus local time it would have to know what
      the timezone setting of the Zaurus is. Given that the desktop software
      does \emph{NOT} know the Zaurus timezone setting the software must use
      the date as passed without converting it. This would make it show up as
      filling the same date as on the Zaurus. However, the times would be off
      by the timezone difference. Most people prefer this over the correct
      way. However, if the developers of the Zaurus synchronization software
      were intelligent they would have passed it over as UTC rather than local
      time so that it could be used in either of the above described manners.

      \item ALSD, ALED - The \emph{ALSD} and \emph{ALED} parameters are used
      to represent the \emph{All Day Start Date} and \emph{All Day End Date}
      of an event. These fields are redundant and contain the same date
      information as the \emph{TIM1} and \emph{TIM2} parameters when an event
      is marked as an \emph{All Day} event. The difference is that the date is
      the only valid portion of these \emph{DATA\_ID\_TIME} parameters. The
      time portion is just random garbage and should be ignored.
    \end{enumerate}

    \section{To-Do Sync Walk Through (After Sync State Obtained)}

    This section consists of a basic walk through of what happens at the
    application layer of the synchronization protocol when one synchronizes
    the To-Do application (initiated from the Desktop). This run down is an
    overview and there for does not contain all the Request and ACK messages
    that would be used in between each of the major messages discussed.

    \subsection{Initiate Synchronization}
    The synchronization is initiated from the Desktop by a client application
    making a TCP connection to the Zaurus on port 4244. As soon as the TCP
    three way hand shake is completed (the connection made) the client
    application sends a \emph{RAY} message to the Zaurus over this
    connection. This \emph{RAY} message is used to tell a server application
    running on the Zaurus that the Desktop has requested to synchronize. This
    server application running on the Zaurus responds to the \emph{RAY}
    message it received by making a TCP connection on port 4245 to a server
    application running on the Desktop. \emph{Note: The original connection
    from the Desktop client application to the Zaurus server application is
    not closed at this point.}

    Now that the connection has been made from the Zaurus to the Desktop
    server application the synchronization can occur. The Desktop server
    responds to receiving a TCP connection from the Zaurus by sending a
    \emph{RAY} message over the newly created TCP connection to the
    Zaurus. The Zaurus after receiving the \emph{RAY} message sends an
    \emph{AAY} message back to the Desktop server.I assume that the \emph{RAY}
    message is requesting to begin the synchronization process and the
    \emph{AAY} message is response of acceptance.

    \subsection{Obtain Authentication State}
    Following the initiation of the synchronization the Desktop server must
    check the Authentication State of the Zaurus to see if it needs to send a
    password to the Zaurus for synchronization to work. The Desktop server
    requests the Authentication State by sending an \emph{RIG} message. The
    Zaurus responds to this request by sending an \emph{AIG} message which
    contains among other things the Authentication State.

    \subsubsection{Authenticate Password}
    If the Authentication State of the Zaurus is one that requires a password
    then the Desktop must send the password to the Zaurus for an
    authentication check. The Desktop server does this by sending an \emph{RRL}
    message, which contains the password, to the Zaurus. If the password is
    successfully authenticated by the Zaurus, the Zaurus responds with an
    \emph{AEX} message.

    \subsection{Check Authentication State}
    Following either the \emph{Authenticate Password} phase or the
    \emph{Obtain Authentication State} phase depending on the value of the
    authentication state received in the \emph{Obtain Authentication State}
    phase, the Desktop server must check to see if the authentication state
    has changed. The Desktop server does this by requesting the authentication
    state from the Zaurus now that a password may have been sent to the Zaurus
    for authentication. It request the authentication state using the same
    method it used in the \emph{Obtain Authentication State}, sending an
    \emph{RIG} message and receiving a \emph{AIG} message in response.

    \subsection{Obtain Zaurus Synchronization State}
    Following checking the authentication the Desktop server application then
    needs to obtain the synchronization state so it can determine if it needs
    to perform a full sync (it sets the sync state) or a item by item sync
    (the sync state is already set). To determine this the Desktop server
    requests the synchronization state from the Zaurus by sending a \emph{RMG}
    message containing the identifier for the type of synchronization to
    perform (To-Do, Calendar, Address, etc). The Zaurus responds to the Desktop
    with a \emph{AMG} message which contains data that notifies the Desktop if
    the Zaurus has a sync state or not.

    \subsection{Request Time of Last Synchronization}
    Following obtaining the synchronization state the Desktop server then
    obtains the time that the last synchronization was performed from the
    Zaurus. The Desktop sends an \emph{RTG} message (request for time of last
    sync)  to the Zaurus. The Zaurus then responds with a \emph{ATG} message
    which contains the time and date of the last synchronization. The Desktop
    then uses this time stamp in comparison with time stamps representing the
    time of object creation for items within its database. Hence, so it will
    know which of its items need to be synchronized.

    \subsection{Setting Time of Current Synchronization}
    After obtaining the time of the last synchronization the Desktop server
    application has enough information to do comparisons, however, the Zaurus
    does not. The Zaurus only has the time of the last synchronization. Hence,
    the Desktop server application takes the current time stamp of the
    computer it is running on and sends that to the Zaurus as the time of the
    current synchronization. The Desktop server applications does this by
    sending a \emph{RTS} message to the Zaurus. The Zaurus then sends a
    response \emph{AEX} message to the Desktop server application providing an
    acceptance of the new synchronization time. The Zaurus records this time
    and will use it as the response data in its \emph{ATG} message the next
    time a synchronization is performed.

    \subsection{Request Sync ID List}
    Once all the synchronization time information has been shared between both
    parties they are able to determine which of their items require
    synchronization. Hence, the Desktop server application requests a list of
    all the Sync IDs associated with in the Zaurus database that need to be
    synchronized. The Desktop server application does this by sending a
    \emph{RSY} message containing a synchronization type identifier of type
    To-Do, Calendar, Address, etc. The Zaurus then responds to the Desktop
    server application by sending it a \emph{ASY} message containing a list of
    all the Sync IDs of items that require synchronization.

    \subsection{Obtain Item Data}
    Since the Desktop server now has a list of Sync IDs of the items it needs
    to obtain the data of from the Zaurus it is ready to synchronize, obtain
    the data and store it in its database. The Desktop server applications
    obtains this data by requesting the data for each Sync ID it found in the
    list that the Zaurus sent to it in the previous section. The Desktop
    server application sends a \emph{RDR} message containing the Sync ID of
    the item it is requesting data for. The Zaurus then responds by sending an
    \emph{ADR} message containing data associated with the item that has that
    Sync ID. The Desktop server does this for every Sync ID it found in the
    list obtained in the above section until all the data has been
    synchronized.

    \subsection{State Finished Synchronizing}
    Now that all the data has been transfered and stored in the Desktop
    server applications database the Desktop wants to notify the Zaurus that
    it is done asking for data to synchronize. The Desktop does this by
    sending a \emph{RDS} message. The Zaurus responds to this by sending an
    \emph{AEX} message declaring acceptance.

    \subsection{Obtain Authentication State}
    Following making the statement of being finished requesting data from the
    Zaurus, the Desktop server application requests the authentication state
    again. The reason for this is not clear due to the fact that the
    authentication state has not changed in between the two messages. The
    Desktop server application does this just as it did in the above sections,
    it sends an \emph{RIG} message to the Zaurus. The Zaurus then responds
    with an \emph{AIG} message which contains the authentication state.

    \subsection{Request Ending of Connection}
    After obtaining the authentication state for the last time the Desktop
    server application requests that the Zaurus close the TCP connection since
    the Zaurus initiated the connection to the Desktop server application. The
    Desktop server application does this by sending an \emph{RQT} to the
    Zaurus. The Zaurus responds to this message by sending an \emph{AEX}
    message stating acceptance and then closing the TCP connection.

    \subsection{Finishing Synchronization}
    Now that the Zaurus has closed the TCP connection to the Desktop server
    application the Desktop client application which initiated the
    synchronization in the first place sends a \emph{Abort Message} (a common
    message) to the Zaurus and then closing the TCP connection to the Zaurus
    server application.

%%% Local Variables: 
%%% mode: latex
%%% TeX-master: t
%%% End: 
